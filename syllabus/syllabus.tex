\documentclass[letterpaper,10pt,onecolumn]{article}
\usepackage[spanish]{babel}
\usepackage[utf8]{inputenc}
\usepackage{amsfonts}
\usepackage{amsthm}
\usepackage{amsmath}
\usepackage{mathrsfs}

\usepackage{enumitem}
\usepackage[pdftex]{color,graphicx}
\usepackage{hyperref}
\usepackage{listings}
\usepackage{calligra}
\usepackage{url}
%\usepackage{algpseudocode} 
\DeclareMathAlphabet{\mathcalligra}{T1}{calligra}{m}{n}
\DeclareFontShape{T1}{calligra}{m}{n}{<->s*[2.2]callig15}{}
\newcommand{\scripty}[1]{\ensuremath{\mathcalligra{#1}}}
\lstloadlanguages{[5.2]Mathematica}
\setlength{\oddsidemargin}{0cm}
\setlength{\textwidth}{490pt}
\setlength{\topmargin}{-40pt}
\addtolength{\hoffset}{-0.3cm}
\addtolength{\textheight}{4cm}

\begin{document}
\begin{center}

\includegraphics[width=490pt]{header.png}\\[0.5cm]

\textsc{\LARGE M\'etodos Computacionales}\\[0.1cm]
\large Jaime E. Forero Romero\\[0.5cm]

\end{center}

\large \noindent\textsc{Nombre del curso:}  M\'etodos Computacionales%Aqui  
                                %nombre del curso 
  
\noindent\textsc{C\'odigo del curso:} FISI 2028 (Magistral) FISI 2809 (Laboratorio) %Aqui el codigo del
                                %curso 

\noindent\textsc{Unidad acad\'emica:} Departamento de F\'isica

\noindent\textsc{Periodo acad\'emico:} 201620 %Aqui el periodo,
                                %p.ej. 201510 

\noindent\textsc{Horario (magistral):} Mi 12:30 a 14:00 y Vi, 15:30 a
17:00 %Aqui el horario, %p.ej. Ma y Ju, 10:00 a 11:20 

\noindent\textsc{Horario (laboratorio):} Ju\

\noindent\rule{\textwidth}{1pt}\\[-0.3cm]

\normalsize \noindent\textsc{Nombre profesor(a) principal:} Jaime
E. Forero Romero%Aqui nombre del profesor principal 

\noindent\textsc{Correo electr\'onico:}
\href{mailto:je.forero@uniandes.edu.co}{\nolinkurl{je.forero@uniandes.edu.co}}
%Cambie address por su direccion de correo uniandes 

\noindent\textsc{Horario y lugar de atenci\'on:} Mi 14:00 a 16:00, Oficina Ip208 
\\[-0.1cm]

\noindent\textsc{Nombre profesor(a) complementario(a):} %Aqui nombre
                                %del profesor complementario si aplica 

\noindent\textsc{Correo electr\'onico:}
\href{mailto:@uniandes.edu.co}{\nolinkurl{@uniandes.edu.co}}

%Cambie address por direccion de correo uniandes del profesor
%complementario 

%\noindent\textsc{Horario y lugar de atenci\'on:} %Aqui horario y
%lugar de atencion del profesor complementario, p.ej. Vi, 15:00 a
%17:00, Oficina Ip102 
%\\[-0.1cm]
%Repetir esto en caso de varios profesores complementarios

%\noindent\textsc{Nombre monitor(a):} %Aqui nombre del monitor si aplica

%\noindent\textsc{Correo electr\'onico:}
%\href{mailto:address@uniandes.edu.co}{\nolinkurl{address@uniandes.edu.co}}
%%Cambie address por direccion de correo uniandes del monitor 

%\noindent\textsc{Horario y lugar de atenci\'on:} %Aqui horario y
%lugar de atencion del monitor, p.ej. Vi, 15:00 a 17:00, Oficina Ip102 

\noindent\rule{\textwidth}{1pt}\\[-0.1cm]

\newcounter{mysection}
\addtocounter{mysection}{1}

\noindent\textbf{\large \Roman{mysection} \quad Introducci\'on}\\[-0.2cm]

%Este espacio es para hacer una introduccion al curso, evidenciando la
%propuesta metodologica. Debe ser clara y precisa. 

\noindent\normalsize Los m\'etodos computacionales son un aspecto
inseparable de cualquier \'area de trabajo en ciencia e ingenier\'ia.
Esto se debe en gran parte a la disminuci\'on en costos y complejidad
de uso de las computadoras programables, unido al aumento
exponencial en su capacidad de procesamiento.
Este curso busca guiar a los estudiantes en el desarrollo de sus
\emph{habilidades computacionales} en generar y procesar
datos para obtener informaci\'on sobre la realidad que esos datos
pretenden  describir.  
Estos datos pueden ser mediciones o simulaciones de
sistemas f\'isicos, qu\'imicos, geol\'ogicos, biol\'ogicos,
financieros o industriales, entre otros.     
El programa del curso tiene dos componentes diferenciadas: m\'etodos
num\'ericos  y \emph{carpinter\'ia} de software.  
La parte de m\'etodos num\'ericos busca ilustrar el paso entre la
formulaci\'on de una pregunta sobre la realidad en t\'erminos
matem\'aticos y su descripci\'on num\'erica/computacional, para
mostrar posibles formas de escribir esa pregunta con software.   
La parte de carpinter\'ia busca presentar algunas t\'ecnicas y
pr\'acticas necesarias (no suficientes) para poder obtener resultados
computacionales reproducibles.   
\\[0.1cm] 

\stepcounter{mysection}
\noindent\textbf{\large \Roman{mysection} \quad Objetivos}\\[-0.2cm]

%En este espacio se debe precisar el ente visor del curso y el
%proposito ideal al finalizar el curso. 
\noindent\normalsize El objetivo principal del curso es presentar
algoritmos y t\'ecnicas b\'asicas para:

\begin{itemize}
\item resolver ecuaciones
  diferenciales ordinarias, parciales y estoc\'asticas, \\[-0.6cm]
\item analizar y describir datos con t\'ecnicas estad\'isticas basadas
  en m\'etodos Monte Carlo, \\[-0.6cm]
\item desarrollar esquemas reproducibles para el an\'alisis de datos cient\'ificos. \\[-0.6cm]
\end{itemize}

\stepcounter{mysection}
\noindent\textbf{\large \Roman{mysection} \quad Competencias a
  desarrollar}\\[-0.2cm] 

%En este espacio se describen las habilidades que el estudiante desarrollara en el transcurso del curso.

\noindent\normalsize Al finalizar el curso, se espera que el
estudiante est\'e en capacidad de: 

\begin{itemize}
\item manejar dos lenguajes de programaci\'on modernos de
  computaci\'on num\'erica: uno de alto
  nivel (Python) y otro de bajo nivel (C),\\[-0.6cm] 
\item implementar algoritmos sencillos para la resoluci\'on de
  ecuaciones diferenciales y para el an\'alisis estad\'istico
  exploratorio de datos, \\[-0.6cm]   
\item desarrollar un esquema sencillo para obtener resultados
  computacionales reproducibles.\\[-0.6cm]  
\end{itemize}

\stepcounter{mysection}
\noindent\textbf{\large \Roman{mysection} \quad Contenido por
  semanas}\\[-0.2cm]  

%Se expone de forma ordenada toda la tematica a tratar del curso. Debe planearse para 15 semanas.

\noindent\textbf{\textsc{Semana 1}}\\[-0.5cm]
\begin{itemize}
\item Temas: 
Presentaci\'on del curso. Unix. Consola. Comandos b\'asicos. Editores
de texto. Int\'erprete de
Python. Variables. Aritm\'etica. Listas. Diccionarios.\\[-0.6cm] 
\item Lecturas preparatorias: \\[-0.6cm]
\item Talleres: Prueba de diagn\'ostico (sin nota). \\[-0.6cm]
\end{itemize}

\noindent\textbf{\textsc{Semana 2}}\\[-0.5cm]
\begin{itemize}
\item Temas: If/while/break/continue. Ciclos. Funciones. Arreglos
(numpy). Lectura y escritura de archivos (numpy). Gr\'aficas y
visualizaci\'on (matplotlib). \\[-0.6cm]
\item Lecturas preparatorias: \\[-0.6cm]
\item Talleres: \\[-0.6cm]
\end{itemize}

\noindent\textbf{\textsc{Semana 3}}\\[-0.5cm]
\begin{itemize}
\item Temas: Integrales y derivadas. \\[-0.6cm]
\item Lecturas preparatorias: \\[-0.6cm]
\item Talleres: \\[-0.6cm]
\end{itemize}

\noindent\textbf{\textsc{Semana 4}}\\[-0.5cm]
\begin{itemize}
\item Temas: Soluci\'on de ecuaciones lineas y no lineales.\\[-0.6cm]
\item Lecturas preparatorias: \\[-0.6cm]
\item Talleres: \\[-0.6cm]
\end{itemize}

\noindent\textbf{\textsc{Semana 5}}\\[-0.5cm]
\begin{itemize}
\item Temas: Autovalores y autovectores. Principal Component
  Analysis. \\[-0.6cm]
\item Lecturas preparatorias: \\[-0.6cm]
\item Talleres: \\[-0.6cm]
\end{itemize}

\noindent\textbf{\textsc{Semana 6}}\\[-0.5cm]
\begin{itemize}
\item Temas: Transformadas de Fourier\\[-0.6cm]
\item Lecturas preparatorias: \\[-0.6cm]
\item Talleres: \\[-0.6cm]
\end{itemize}

\noindent\textbf{\textsc{Semana 7}}\\[-0.5cm]
\begin{itemize}
\item Temas: Ecuaciones diferenciales ordinarias. \\[-0.6cm]
\item Lecturas preparatorias: \\[-0.6cm]
\item Talleres: \\[-0.6cm]
\end{itemize}

\noindent\textbf{\textsc{Semana 8}}\\[-0.5cm]
\begin{itemize}
\item Temas: Ecuaciones diferenciales parciales. \\[-0.6cm]
\item Lecturas preparatorias: \\[-0.6cm]
\item Talleres: \\[-0.6cm]
\end{itemize}

%%% SEMANA DE TRABAJO INDIVIDUAL EN 201620

\noindent\textbf{\textsc{Semana 9}}\\[-0.5cm]
\begin{itemize}
\item Temas: C. \\[-0.6cm]
\item Lecturas preparatorias: \\[-0.6cm]
\item Talleres: \\[-0.6cm]
\end{itemize}

\noindent\textbf{\textsc{Semana 10}}\\[-0.5cm]
\begin{itemize}
\item Temas: C. Git. Github. \\[-0.6cm]
\item Lecturas preparatorias: \\[-0.6cm]
\item Talleres: \\[-0.6cm]
\end{itemize}

\noindent\textbf{\textsc{Semana 11}}\\[-0.5cm]
\begin{itemize}
\item Temas: Makefile. Unit Tests. \\[-0.6cm]
\item Lecturas preparatorias: \\[-0.6cm]
\item Talleres: \\[-0.6cm]
\end{itemize}


\noindent\textbf{\textsc{Semana 12}}\\[-0.5cm]
\begin{itemize}
\item Temas: Conceptos b\'asicos de
  probabilidad y estad\'istica. Procesos aleatorios. \\[-0.6cm]  
\item Lecturas preparatorias: \\[-0.6cm]
\item Talleres: \\[-0.6cm]
\end{itemize}

\noindent\textbf{\textsc{Semana 13}}\\[-0.5cm]
\begin{itemize}
\item Temas: M\'etodos de Monte Carlo.\\[-0.6cm]
\item Lecturas preparatorias: \\[-0.6cm]
\item Talleres: \\[-0.6cm]
\end{itemize}

\noindent\textbf{\textsc{Semana 14}}\\[-0.5cm]
\begin{itemize}
\item Temas: Estimaci\'on de par\'ametros en estad\'istica
  bayesiana.\\[-0.6cm] 
\item Lecturas preparatorias: \\[-0.6cm]
\item Talleres: \\[-0.6cm]
\end{itemize}

\noindent\textbf{\textsc{Semana 15}}\\[-0.5cm]
\begin{itemize}
\item Temas: Ecuaciones diferenciales estoc\'asticas. \\[-0.6cm]
\item Lecturas preparatorias: \\[-0.6cm]
\item Talleres: \\[-0.6cm]
\end{itemize}

\stepcounter{mysection}
\noindent\textbf{\large \Roman{mysection} \quad
  Metodolog\'ia}\\[-0.2cm] 

%Se describen las tecnicas y metodos para el desarrollo exitoso del curso.

\noindent\normalsize Cada clase tendr\'a una corta presentaci\'on
te\'orica (30 minutos aproximadamente) para pasar a practicar todos
los conceptos directamente en la computadora/cluster a trav\'es de
ejercicios de pr\'actica (50 minutos aproximadamente). \\[0.1cm]


\stepcounter{mysection}
\noindent\textbf{\large \Roman{mysection} \quad Criterios de
  evaluaci\'on}\\[-0.2cm] 


El curso tendr\'a tres entregas de trabajos, cada una con un valor
del $(100/3) \%$ de la nota definitiva. Los temas de las entregas ser\'an
los siguientes:
\begin{enumerate}
\item Ecuaciones diferenciales (ordinarias, parciales, estoc\'asticas).
\\[-0.6cm]
\item Machine Learning.
\\[-0.6cm]
\item C\'omputo en paralelo (OpenMP, MPI).
\\[-0.2cm]
\end{enumerate}


\stepcounter{mysection}
\noindent\textbf{\large \Roman{mysection} \quad
  Bibliograf\'ia}\\[-0.2cm] 

%Indicar los libros y la documentacion guia.

\noindent\normalsize Bibliograf\'ia principal:

\begin{itemize}
\item R. L. Burden, J. D. Faires. \textit{Numerical analysis},
  2011. (Biblioteca General - 519.4 B862 2011)\\[-0.6cm]
\item A. Tveito, H.P. Langtangem, B.F. Nielsen., \textit{Elements of
  Scientific Computing}, 2010.  (Biblioteca General, Recurso
  Electr\'onico 510. )\\[-0.6cm] 
\item O. Maimon and L. Rokach, \textit{The Data Mining and Knowledge
  Discovery Handbook}, 2010. (Biblioteca General, Recurso
  Electr\'onico 006.312)\\[-0.6cm]
\item M. Snir, \textit{MPI : the complete reference},
  1996. (Biblioteca General, 004.35 M637)\\[-0.6cm]
\item J. Sanders, E. Kandrot. \textit{CUDA by example: an
  introduction to general-purpose GPU programming}, 2010. (Biblioteca
  General - 005.275 S152)\\[-0.2cm]
\end{itemize} 

\noindent\normalsize Bibliograf\'ia complementaria:

\begin{itemize}
\item D. Conway and J. M. White. \textit{Machine learning for
    hackers}, 2012.\\[-0.6cm]
\item S.Bird. \textit{Natural Language Processing with
  Python}, 2009.\\[-0.6cm]
\item Theano Development. \textit{Deep Learning Tutorial}
  \url{http://deeplearning.net/tutorial/}  \\[-0.2cm]
\end{itemize}


\end{document}



\noindent
\textbf{FISI 2028, 2029 Métodos Computacionales (Magistral y Laboratorio)} 
Semestre 201620\\

%\\


\section*{Objetivo}
El curso tiene como objetivo principal guiar a los estudiantes en el
desarrollo de sus \emph{habilidades computacionales}.
Nos interesan las habilitades \'utiles para trabajar con computadores
en generar y procesar datos para obtener informaci\'on sobre la
realidad que esos datos pretenden  describir. 
Estos datos pueden ser mediciones o simulaciones de
sistemas f\'isicos, qu\'imicos, geol\'ogicos, biol\'ogicos,
financieros o industriales, entre otros.     
En este curso, estas habilidades incluyen el aprendizaje de al menos
dos lenguajes de programaci\'on, la familiarizaci\'on con diferentes
m\'etodos num\'ericos, la  implementaci\'on de t\'ecnicas de
desarrollo de software y la pr\'actica de una actitud esc\'eptica ante
resultados computacionales. 

\section*{Metodolog\'ia}
Las \emph{habilidades computacionales} se desarrollan trabajando activamente. 
Por esto en las sesiones magistrales, luego de presentar un resumen de
los conceptos te\'oricos, se har\'a \'enfasis en la pr\'actica y experimentaci\'on.  
Para que esto funcione es necesario que los estudiantes estudien el
tema correspondiente {\bf antes de cada clase}.

El programa del curso tiene dos componentes diferenciadas. 
La parte de m\'etodos num\'ericos  y la parte de
\emph{carpinter\'ia} de software. 
La parte de m\'etodos num\'ericos ilustra como pasar de alg\'una
pregunta sobre la realidad a un formalismo matem\'atico general, luego
a una descripci\'on num\'erica y de ah\'i a su implementaci\'on en
t\'erminos de software. 
La parte de carpinter\'ia busca presentar algunas condiciones
m\'inimas para que los resultados del \'ultimo paso sea confiables y
permitan la producci\'on de resultados reproducibles.

En el \emph{Laboratorio de M\'etodos
Computationales} habr\'a m\'as tiempo para practicar lo visto en la
clase magistral, hacer ejercicios y aclarar dudas.  

\section*{Lenguajes}
\noindent Utilizaremos dos lenguajes de programaci\'on:
C y Python. Eventualmente utilizaremos notebooks de Jupyter para
trabajar con Python.
No se  aceptar\'an tareas en Matlab, Mathematica, R, C++ o cualquier
otro lenguaje  de programaci\'on.

\section*{Evaluaci\'on - Clase Magistral}


Al comienzo del semestre se har\'a un examen (corto, sin nota) para
diagnosticar el conocimiento general que ya tienen los estudiantes
sobre los temas del curso. 

Durante el cruso dar\'an cinco talleres y un bono para resolver cada
uno con un valor del $20\%$ de la nota definitiva.  
\textbf{Para poder entregar el bono es necesario responder un examen
  escrito (corto) sobre los temas te\'oricos vistos durante el semestre.}



Todos los talleres ser\'an \textbf{individuales}. 
Si en las entregas individuales se detecta que el trabajo no fue
individual (esto incluye colaboraci\'on con personas no inscritas en
el curso), entonces la nota de todos los talleres quedar\'a
autom\'aticamente {\bf en cero}.

Todas las entregas se har\'an a trav\'es de SICUA. 
{\bf No se aceptar\'a ninguna tarea por fuera de esa plataforma}, a
menos que ocurra un una falla en los servidores de SICUA que afecte a
{\bf todos} los estudiantes del curso. 

De acuerdo a la nota definiva en Laboratorio habr\'a {\bf otro bono} en la
nota definitiva de la Clase Magistral. 
Siendo $x$ la nota de Laboratorio, el bono correspondiente
se calcula as\'i:
$4.0 < x \leq 4.4 \rightarrow 0.1$, $4.4< x\leq 4.8\rightarrow 0.2$, $4.8<x
\leq 5.0\rightarrow 0.4$.


El curso cuenta con un repositorio en GitHub:

\url{https://github.com/ComputoCienciasUniandes/MetodosComputacionales}. 

El material se
encuentra distribuido en las siguientes carpetas. 


\begin{itemize}
\item \texttt{hands\_on/}: Ejemplos para hacer en clase.
\item \texttt{homework/}: Enunciados y calificaciones de las tareas.
\item \texttt{notas/}: Notas de clase.
\item \texttt{syllabus/}: Programa del curso.

\end{itemize}


\section*{Evaluaci\'on - Laboratorio}

En las sesiones de Laboratorio se plantear\'an un total de cinco
ejercicios cortos para desarrollar y entregar en clase.
Cada ejercicio corresponde a un $20\%$ de la nota final.
El repositorio de GitHub del Laboratorio es:

  \url{https://github.com/ComputoCienciasUniandes/MetodosComputacionalesLaboratorio}. 


Todas las entregas se har\'an a trav\'es de SICUA. 
{\bf No se aceptar\'a ninguna tarea por fuera de esa plataforma} a
menos que ocurra un una falla en los servidores de SICUA que afecte a
{\bf todos} los estudiantes del curso. 



 
 
\newpage
\section*{Programa}

\begin{center}
\renewcommand{\arraystretch}{1.1}
\begin{tabular}{|p{1.6cm}|p{4.5cm}|p{4.50cm}|p{2.0cm}|p{2.0cm}|}
\hline
Semana & Teor\'ia & Carpinter\'ia & Taller (Mag.) & Ejercicio (Lab.) \\\hline
1 (18.1) &     & Linux / Consola / Editores de texto &  Ex\'amen
diagn\'ostico (sin nota) & \\\hline
2 (25.1) & 	& Python b\'asico & & \#1 \\\hline
3 (1.2) & 	& Python (objetos, numpy, matplotlib), Jupyter& \#1 & \\\hline 
4 (8.2) & 	Operaciones matriciales, sistemas de ecuaciones lineales,
m\'inimos cuadrados & &  & \\\hline
5 (15.2) &  Autovalores, autovectores, PCA y tensores& & & \# 2 \\\hline
6 (22.2) &  Interpolaci\'on, extraploaci\'on y b\'usqueda de ra\'ices
& & \#2 & \\\hline
7 (29.2) &  Transformada de Fourier & & & \\ \hline
8 (7.3) & Derivaci\'on e integraci\'on &  &  & \#3 \\\hline
9 (21.3) & {\bf Semana de trabajo individual} & & \# 3&\\\hline
10 (14.3) & & Git, GitHub &  & \\\hline
11 (28.3) & & C b\'asico & & \#4 \\\hline
14 (18.4) & & Makefiles y Testing & \#4& \\\hline
12 (4.4) & Ecuaciones diferenciales ordinarias & & & \\\hline 
13 (11.4) & Ecuaciones diferenciales parciales & & & \#5 \\\hline
15 (25.4) & Estad\'istica frecuentista. Estad\'istica Bayesiana. &   &
\#5  & \\\hline 
16 (2.5) & Markov Chain Monte Carlo&   & Ex\'amen y
Bono &\\\hline 
\end{tabular}
\end{center}

\section*{Lecturas/Actividades requeridas para cada semana.}

\begin{enumerate}
\item A
\item B

\end{enumerate}

\section*{Referencias Bibliogr\'aficas}

\begin{itemize}
\item
\textit{Elements of Scientific Computing}
Tveito A., Langtangen H.P., Nielsen B.F., Cai X. Spinger. 2010.
\item
\textit{A survey of Computational Physics}
. R. H. Landau, M. J. P\'aez, C. C. Bordeianu. Princeton Univ. Press. 2006
\item 
\textit{Statistical Mechanics: Algorithms and Computations.}
W. Krauth, Oxford Univ. Press. 
\item 
\textit{Introduction to Computation and Programming Using Python},
Guttag, J. V. The MIT Press. 2013.
\item 
\textit{The C programming language.}
 B. Kernighan \& D. Ritchie, Second Edition, Prentice Hall.
\item\url{http://software-carpentry.org/}
\item\url{http://people.ds.cam.ac.uk/nmm1/Fortran/index.html}
\item\url{http://xkcd.com/}
\item\url{https://www.khanacademy.org}
\end{itemize}

 

\end{document}
