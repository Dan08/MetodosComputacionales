% Metodos Computacionales Syllabus
\documentclass[11pt]{article}
\usepackage[utf8]{inputenc}
\usepackage[intlimits]{amsmath}
\usepackage{hyperref}
\usepackage{amsfonts}
\usepackage{amscd}
\usepackage{amssymb}
\usepackage{natbib}
\usepackage[spanish]{babel}
\usepackage[pdftex]{color,graphicx}
\textheight=25.5cm
\textwidth=16.0cm
\oddsidemargin=-0.5cm
\topmargin=-2.0cm
%\usepackage[latin1]{inputenc}
%\usepackage[utf8]{inputenc}



\begin{document}

\includegraphics[width=490pt]{header.png}\\[0.5cm]

\noindent
\textbf{FISI 2028, 2029 Métodos Computacionales y Laboratorio} Semestre 201620\\
\textbf{Magistral} - Martes y Jueves 7:00 - 8:30 am \\
\textbf{Laboratorio} - Mi\'ercoles 1:00 - 2:30 pm \\
Salón Q508\\
Profesor: Jaime Forero, email: \textbf{je.orero}\\
Oficina, Monitores: Por definir\\
Horario de Atención: Con cita previa
%\\


\section*{Objetivo}
El curso tiene como objetivo principal desarrollar los fundamentos 
de una \emph{actitud computacional}.  
Esta \emph{actitud } corresponde a un conjunto de habilidades para
trabajar con computadores en generar y procesar datos para obtener
informaci\'on sobre la realidad que esos datos pretenden
describir. 
Estos datos pueden corresponder con mediciones o simulaciones de
sistemas f\'isicos, geol\'ogicos, biol\'ogicos, financieros o
industriales, entre otros.     
Estas habilidades incluyen el aprendizaje de al menos dos lenguajes de
programaci\'on, la familiarizaci\'on con diferentes algoritmos, la
implementaci\'on de t\'ecnicas de desarrollo de software y la
pr\'actica de una actitud esc\'eptica ante resultados
computacionales.

\section*{Metodolog\'ia}
Esa \emph{actitud computacional} se desarrolla trabajando. 
Las sesiones magistrales de este curso estar\'an enfocadas
en la exploraci\'on, pr\'actica y la experimentaci\'on. 
Para que esto  funcione es necesario que los estudiantes estudien el
tema correspondiente {\bf antes de cada clase}.  

El programa del curso tiene dos componentes diferenciadas. La parte de
m\'etodos de computo num\'erico y la parte de
\emph{carpinter\'ia} de software. 
La parte de m\'etodos num\'ericos ilustra como pasar de alg\'una
pregunta sobre la realidad a un formalismo
matem\'atico general, luego a una descripci\'on num\'erica y de ah\'i
a su implementaci\'on en t\'erminos de software.
La parte de carpinter\'ia busca presentar algunas pr\'acticas
necesarias para el desarrollo de software de calidad.

Esta materia se ve al mismo tiempo que el \emph{Laboratorio de M\'etodos
Computationales}. 
El objetivo del Laboratorio es tener m\'as tiempo para practicar lo
visto en clase, hacer ejercicios y aclarar dudas.

\section*{Lenguajes}
\noindent Se usar\'an principalmente doe lenguajes de programaci\'on:
C y Python. Eventualmente utilizaremos notebooks de IPython.
No se  aceptar\'an tareas en Matlab, Mathematica, R, C++ o cualquier
otro lenguaje  de programaci\'on que no este en la lista mencionada
antes.  

\section*{Evaluaci\'on - Laboratorio}

En las sesiones de Laboratorio se dejar\'an talleres cada 15 d\'ias. El repositorio de GitHub del Laboratorio es

  \url{https://github.com/ComputoCienciasUniandes/MetodosComputacionalesLaboratorio}. La nota del Laboratorio será el promedio de las notas de los talleres entregados.

\section*{Evaluaci\'on - Clase Magistral}

Hay 7 talleres para entregar. \textbf{No habr\'a parciales ni
examen final}. Todos los talleres ser\'an \textbf{individuales}. 
Si en las entregas
individuales se detecta que hubo trabajo en grupos entonces la nota de
todos los talleres quedar\'a autom\'aticamente en cero
\textbf{(0.0)}.  

Las entregas para los \'ultimos 3 talleres se har\'an en dos tiempos:
una primera entrega donde se muestre expl\'icitamente un borrador del
c\'odigo con comentarios, luego la entrega definitiva con el c\'odigo
completo. La primera entrega es una condici\'on necesaria para aceptar
la segunda. Solamente la segunda entrega recibe una nota. 


De acuerdo a la
nota definiva en Laboratorio habr\'a un bono en la nota definitiva de
la Clase Magistral. Siendo $x$ la nota de Laboratorio, el bono correspondiente
se calcula as\'i:
$4.0 < x \leq 4.4 \rightarrow 0.1$, $4.4< x\leq 4.8\rightarrow 0.2$, $4.8<x
\leq 5.0\rightarrow 0.5$.


El curso cuenta con un repositorio en GitHub:

\url{https://github.com/ComputoCienciasUniandes/MetodosComputacionales}. 

El material se
encuentra distribuido en las siguientes carpetas. 


\begin{itemize}
\item \texttt{hands\_on/}: Ejemplos para hacer en clase.
\item \texttt{homework/}: Enunciados y calificaciones de las tareas.
\item \texttt{notas/}: Notas de clase.
\item \texttt{syllabus/}: Programa del curso.

\end{itemize}
 
 

\section*{Programa}

\begin{center}
\renewcommand{\arraystretch}{1.1}
\begin{tabular}{|p{1.6cm}|p{7.0cm}|p{4.0cm}|p{2.5cm}|}
\hline

Semana & Teor\'ia & Carpinter\'ia & Taller \\\hline
1 (18.1) &     & Linux / Consola / Editores de texto & \\\hline
2 (25.1) & 	& Python b\'asico & \# 1 (15\%) \\\hline
3 (1.2) & 	& Python (objetos, numpy, matplotlib), IPython& \\\hline 
4 (8.2) & 	Operaciones matriciales, sistemas de ecuaciones lineales,
m\'inimos cuadrados & &  \# 2 (15\%) \\\hline
5 (15.2) &  Autovalores, autovectores, PCA y tensores& & \\\hline
6 (22.2) &  Interpolaci\'on, extraploaci\'on y b\'usqueda de ra\'ices & & \#
3 (15\%)\\\hline
7 (29.2) &  Transformada de Fourier & & \\ \hline
8 (7.3) & Derivaci\'on e integraci\'on &  & \# 4 (15 \%)\\\hline
9 (14.3) & & Git, GitHub &  \\\hline
10 (21.3) & {\bf Semana de trabajo individual} & &\\\hline
11 (28.3) & & Lenguajes compilados, makefiles & \#5 (10\%) \\\hline
12 (4.4) & Ecuaciones diferenciales ordinarias (1er orden)& & \\\hline 
13 (11.4) & Ecuaciones diferenciales ordinares (2do orden)& & \#6 (15 \%)\\\hline
14 (18.4) & Ecuaciones diferenciales parciales &  & \\\hline
15 (25.4) & Markov Chain Monte Carlo &   & \#7 (15\%) \\\hline
16 (2.5) & C\'omputo paralelo &    & \\\hline
\hline
\end{tabular}
\end{center}

\newpage

\section*{Referencias Bibliogr\'aficas}

\begin{itemize}
\item
\textit{Elements of Scientific Computing}
Tveito A., Langtangen H.P., Nielsen B.F., Cai X. Spinger. 2010.
\item
\textit{A survey of Computational Physics}
. R. H. Landau, M. J. P\'aez, C. C. Bordeianu. Princeton Univ. Press. 2006
\item 
\textit{Statistical Mechanics: Algorithms and Computations.}
W. Krauth, Oxford Univ. Press. 
\item 
\textit{Introduction to Computation and Programming Using Python},
Guttag, J. V. The MIT Press. 2013.
\item 
\textit{The C programming language.}
 B. Kernighan \& D. Ritchie, Second Edition, Prentice Hall.
\item\url{http://software-carpentry.org/}
\item\url{http://people.ds.cam.ac.uk/nmm1/Fortran/index.html}
\item\url{http://xkcd.com/}
\item\url{https://www.khanacademy.org}
\end{itemize}

 

\end{document}
