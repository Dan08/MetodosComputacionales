% Metodos Computacionales Syllabus
\documentclass[11pt]{article}
\usepackage[utf8]{inputenc}
\usepackage[intlimits]{amsmath}
\usepackage{hyperref}
\usepackage{amsfonts}
\usepackage{amscd}
\usepackage{amssymb}
\usepackage{natbib}
\usepackage[spanish]{babel}
\usepackage[pdftex]{color,graphicx}
\textheight=25.5cm
\textwidth=16.0cm
\oddsidemargin=-0.5cm
\topmargin=-2.0cm
%\usepackage[latin1]{inputenc}
%\usepackage[utf8]{inputenc}



\begin{document}

\includegraphics[width=490pt]{header.png}\\[0.5cm]

\noindent
\textbf{FISI 2028, 2029 Métodos Computacionales (Magistral y Laboratorio)} 
Semestre 201620\\
\textbf{Magistral} - Martes y Jueves 7:00 - 8:30 am \\
\textbf{Laboratorio} - Mi\'ercoles 1:00 - 2:30 pm \\
Salón Q508\\
Profesor: Jaime Forero, email: \textbf{je.forero}\\
Oficina, Monitores: Por definir\\
Horario de Atención: Con cita previa
%\\


\section*{Objetivo}
El curso tiene como objetivo principal desarrollar los fundamentos 
de una buena \emph{actitud computacional}.  
Esta buena \emph{actitud} corresponde a un conjunto de habilidades para
trabajar con computadores en generar y procesar datos para obtener
informaci\'on sobre la realidad que esos datos pretenden
describir. 
Estos datos pueden ser mediciones o simulaciones de
sistemas f\'isicos, geol\'ogicos, biol\'ogicos, financieros o
industriales, entre otros.     
Estas habilidades incluyen el aprendizaje de al menos dos lenguajes de
programaci\'on, la familiarizaci\'on con diferentes m\'etodos num\'ericos, la
implementaci\'on de t\'ecnicas de desarrollo de software y la pr\'actica de una actitud esc\'eptica ante resultados computacionales.

\section*{Metodolog\'ia}
Esa \emph{actitud computacional} se desarrolla trabajando. 
Las sesiones magistrales, luego de presentar un resumen de los conceptos te\'oricos, se har\'a \'enfasis en la pr\'actica y experimentaci\'on. 
Para que esto  funcione es necesario que los estudiantes estudien el
tema correspondiente {\bf antes de cada clase}.  

El programa del curso tiene dos componentes diferenciadas. La parte de
m\'etodos de computo num\'erico y la parte de
\emph{carpinter\'ia} de software. 
La parte de m\'etodos num\'ericos ilustra como pasar de alg\'una
pregunta sobre la realidad a un formalismo
matem\'atico general, luego a una descripci\'on num\'erica y de ah\'i
a su implementaci\'on en t\'erminos de software.
La parte de carpinter\'ia busca presentar algunas pr\'acticas
necesarias para el desarrollo de software de calidad.

Esta materia se ve al mismo tiempo que el \emph{Laboratorio de M\'etodos
Computationales}. 
El objetivo del Laboratorio es tener m\'as tiempo para practicar lo
visto en clase, hacer ejercicios y aclarar dudas.

\section*{Lenguajes}
\noindent Se usar\'an principalmente doe lenguajes de programaci\'on:
C y Python. Eventualmente utilizaremos notebooks de Jupyter.
No se  aceptar\'an tareas en Matlab, Mathematica, R, C++ o cualquier
otro lenguaje  de programaci\'on que no este en la lista mencionada
antes.  

\section*{Evaluaci\'on - Laboratorio}

En las sesiones de Laboratorio se plantear\'a un total de cinco
ejercicios cortos para desarrollar y entregar en clase.
Cada ejercicio corresponde a un $20\%$ de la nota final.
El repositorio de GitHub del Laboratorio es:

  \url{https://github.com/ComputoCienciasUniandes/MetodosComputacionalesLaboratorio}. 


Todas las entregas se har\'an a trav\'es de SICUA. 
{\bf No se aceptar\'a ninguna tarea por fuera de esa plataforma} a
menos que ocurra un una falla en los servidores de SICUA que afecte a
{\bf todos} los estudiantes del curso. 



\section*{Evaluaci\'on - Clase Magistral}

Se dar\'an cinco talleres y un bono para resolver y calificar, cada uno
con un valor del $20\%$ de la nota definitiva. 
\textbf{Para poder entregar el bono es necesario responder un examen
  escrito (corto) sobre los temas te\'oricos vistos durante el semestre.}. 
Al comienzo del semestre se har\'a un examen (sin nota) para
diagnosticar el conocimiento general que ya tienen los estudiantes
sobre los temas del curso.


Todos los talleres ser\'an \textbf{individuales}. 
Si en las entregas individuales se detecta que el trabajo no fue
individual (esto incluye colaboraci\'on con personas no inscritas en
el curso), entonces la nota de todos los talleres quedar\'a
autom\'aticamente {\bf en cero}.

Todas las entregas se har\'an a trav\'es de SICUA. 
{\bf No se aceptar\'a ninguna tarea por fuera de esa plataforma}, a
menos que ocurra un una falla en los servidores de SICUA que afecte a
{\bf todos} los estudiantes del curso. 

De acuerdo a la nota definiva en Laboratorio habr\'a {\bf otro bono} en la
nota definitiva de la Clase Magistral. 
Siendo $x$ la nota de Laboratorio, el bono correspondiente
se calcula as\'i:
$4.0 < x \leq 4.4 \rightarrow 0.1$, $4.4< x\leq 4.8\rightarrow 0.2$, $4.8<x
\leq 5.0\rightarrow 0.4$.


El curso cuenta con un repositorio en GitHub:

\url{https://github.com/ComputoCienciasUniandes/MetodosComputacionales}. 

El material se
encuentra distribuido en las siguientes carpetas. 


\begin{itemize}
\item \texttt{hands\_on/}: Ejemplos para hacer en clase.
\item \texttt{homework/}: Enunciados y calificaciones de las tareas.
\item \texttt{notas/}: Notas de clase.
\item \texttt{syllabus/}: Programa del curso.

\end{itemize}
 
 
\newpage
\section*{Programa}

\begin{center}
\renewcommand{\arraystretch}{1.1}
\begin{tabular}{|p{1.6cm}|p{4.5cm}|p{4.50cm}|p{2.0cm}|p{2.0cm}|}
\hline
Semana & Teor\'ia & Carpinter\'ia & Taller (Mag.) & Ejercicio (Lab.) \\\hline
1 (18.1) &     & Linux / Consola / Editores de texto &  Ex\'amen
diagn\'ostico (sin nota) & \\\hline
2 (25.1) & 	& Python b\'asico & & \#1 \\\hline
3 (1.2) & 	& Python (objetos, numpy, matplotlib), Jupyter& \#1 & \\\hline 
4 (8.2) & 	Operaciones matriciales, sistemas de ecuaciones lineales,
m\'inimos cuadrados & &  & \\\hline
5 (15.2) &  Autovalores, autovectores, PCA y tensores& & & \# 2 \\\hline
6 (22.2) &  Interpolaci\'on, extraploaci\'on y b\'usqueda de ra\'ices
& & \#2 & \\\hline
7 (29.2) &  Transformada de Fourier & & & \\ \hline
8 (7.3) & Derivaci\'on e integraci\'on &  &  & \#3 \\\hline
9 (21.3) & {\bf Semana de trabajo individual} & & \# 3&\\\hline
10 (14.3) & & Git, GitHub &  & \\\hline
11 (28.3) & & C b\'asico & & \#4 \\\hline
14 (18.4) & & Makefiles y Testing & \#4& \\\hline
12 (4.4) & Ecuaciones diferenciales ordinarias & & & \\\hline 
13 (11.4) & Ecuaciones diferenciales parciales & & & \#5 \\\hline
15 (25.4) & Estad\'istica frecuentista. Estad\'istica Bayesiana. &   &
\#5  & \\\hline 
16 (2.5) & Markov Chain Monte Carlo&   & Ex\'amen y
Bono &\\\hline 
\end{tabular}
\end{center}

\section*{Lecturas/Actividades requeridas para cada semana.}

\begin{enumerate}
\item A
\item B

\end{enumerate}

\section*{Referencias Bibliogr\'aficas}

\begin{itemize}
\item
\textit{Elements of Scientific Computing}
Tveito A., Langtangen H.P., Nielsen B.F., Cai X. Spinger. 2010.
\item
\textit{A survey of Computational Physics}
. R. H. Landau, M. J. P\'aez, C. C. Bordeianu. Princeton Univ. Press. 2006
\item 
\textit{Statistical Mechanics: Algorithms and Computations.}
W. Krauth, Oxford Univ. Press. 
\item 
\textit{Introduction to Computation and Programming Using Python},
Guttag, J. V. The MIT Press. 2013.
\item 
\textit{The C programming language.}
 B. Kernighan \& D. Ritchie, Second Edition, Prentice Hall.
\item\url{http://software-carpentry.org/}
\item\url{http://people.ds.cam.ac.uk/nmm1/Fortran/index.html}
\item\url{http://xkcd.com/}
\item\url{https://www.khanacademy.org}
\end{itemize}

 

\end{document}
