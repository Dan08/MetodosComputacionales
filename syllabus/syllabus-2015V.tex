% Metodos Computacionales Syllabus
% Curso vacaciones 2015
\documentclass[letterpaper,10pt,onecolumn]{article}
\usepackage[utf8]{inputenc}
\usepackage[intlimits]{amsmath}
\usepackage{hyperref}
\usepackage{amsfonts}
\usepackage{amscd}
\usepackage{amssymb}
\usepackage{natbib}
\usepackage[spanish]{babel}
\usepackage[absolute]{textpos} 
\usepackage[pdftex]{color,graphicx}
\setlength{\oddsidemargin}{0cm}
\setlength{\textwidth}{490pt}
\setlength{\textheight}{680pt}
\setlength{\topmargin}{-40pt}
\addtolength{\hoffset}{-0.3cm}
\usepackage{pifont}
\usepackage{xcolor}

\hypersetup{%
  colorlinks=true,% hyperlinks will be coloured
  urlcolor=blue
}

\begin{document}
\thispagestyle{empty}
\begin{flushleft}

%\includegraphics[width=490pt]{header.png}\\[0.5cm]

\textsc{\huge Métodos Computacionales}\\[0.01cm]

%\Large \textsc{Programa}\\[0.1cm]
\normalsize (FISI 2028) \\
Curso de vacaciones 2015

\end{flushleft}

		\begin{textblock*}{20mm}(140mm,15mm)
  			\includegraphics[height=30 mm]{andes.png}
		\end{textblock*}

\noindent
Profesor: Juan David Lizarazo, email: \textbf{jd.lizarazo10}\\
Oficina: I113 // Horario de atención: martes de 8 a 10 AM. \\ 
\vspace{0.3cm}

\hfill\begin{minipage}{\dimexpr\textwidth-3cm}
\begin{flushright}\textit{``Computers are incredibly fast, accurate and stupid. On the other hand, a well trained operator as compared with a computer is incredibly slow, inaccurate and brilliant.'' Together they are powerful beyond imagination.}\end{flushright}
\end{minipage}

\noindent \\[-0.5cm]
\begin{flushright}\textsc{$\approx$ Leo Cherne - 1969}\end{flushright}

\vspace{0.1cm}

En nuestro curso queremos aprender métodos para resolver diferentes problemas numéricos e implementarlos en algún lenguaje de programación para resolver problemas de interés. En ocasiones vamos a tomar datos del mundo ``real'' y en otras del mundo simulado pero siempre vamos a estar interesados en las formas de procesarlos y visualizarlos, y en extraer información de ellos. Para ello vamos a refinar las herramientas computacionales que ya conocemos y a aprender otras nuevas: un poco más de \verb+Python+ y \verb+Bash+, GitHub para colaborar al estilo moderno, y algo de \verb+C+ entre otras. Ayudados de nuevas herramientas vamos a explorar el interesante mundo que se pone a nuestro alcance: desde los misterios de $\pi$, pasando por la fabulosa internet y hasta el movimiento del sistema solar.

\vspace{0.3cm}
\noindent El curso va a evaluarse a partir de tres items: talleres (60\%), exámenes cortos (25\%) y un proyecto final (15\%). Los talleres cada semana y los exámenes cada dos.

\noindent Para coordinar nuestras actividades contamos con \verb+Sicua+ y con un \href{https://github.com/ComputoCienciasUniandes/MetodosComputacionales}{repositorio} en GitHub.

% Este curso se ve al mismo tiempo que el \emph{Laboratorio de Métodos Computationales}. El objetivo del laboratorio es tener más tiempo para practicar todos lo visto en clase. 
\vspace{0.5cm}

\noindent\textbf{Semana 1:} \verb+bash+, \verb+wget+, \verb+curl+, \verb+awk+, \verb+sed+, \verb+grep+, \verb+sort+, expresiones regulares, \verb+ssh+, \verb+sftp+, \verb+vi+.

\noindent\textbf{Semana 2:} C, \verb+gcc+, Python, Markdown, testing and debugging, complejidad computacional, manejo de excepciones, \verb+make+, error e incertidumbre en cálculos numéricos.

\noindent\textbf{Semana 3:} algebra lineal, \verb+git+, GitHub individual, IPython, herramientas de visualización.

\noindent\textbf{Semana 4:} interpolación, testing and debugging, GitHub en colaboración, análisis de Fourier, procesamiento de imágenes.

\noindent\textbf{Semana 5:} integración y diferenciación numéricas, ecuaciones diferenciales ordinarias (1er orden), NumPy.

\noindent\textbf{Semana 6:} ecuaciones diferenciales ordinarias (2do orden).

\noindent\textbf{Semana 7:} ecuaciones diferenciales parciales.

\noindent\textbf{Semana 8:} métodos Monte Carlo, generación de números pseudoaleatorios.

\noindent\textbf{Semana 9:} proyecto final.

\vspace{0.1cm}
\noindent\textbf{Temas opcionales:} R, Machine Learning, métodos de optimización, dinámica molecular, geometría computacional, uso del clúster, Zotero, Pandas, procesamiento en paralelo.

\section*{Bibliografía}

\begin{itemize}
\item  R. Landau, M. Paez, C. Bordeianu. \href{http://www.compadre.org/psrc/items/detail.cfm?ID=11578}{A Survey of Computational Physics: Python Multimodal eBook}, 2012.
%\ding{168} \ding{171}
\item  J. Guttag. \textit{Introduction to Computation and Programming Using Python}, 2013.
\item  A. Tveito, H. Langtangen, B. Nielsen, X. Cai  \href{http://link.springer.com.ezproxy.uniandes.edu.co:8080/book/10.1007\%2F978-3-642-11299-7}{Elements of Scientific Computing}, 2010.
\item S. Chacon, B. Straub. \href{http://link.springer.com.ezproxy.uniandes.edu.co:8080/book/10.1007\%2F978-1-4302-1834-0}{Pro Git}, 2014.
\item B. Stephenson. \href{http://link.springer.com.ezproxy.uniandes.edu.co:8080/book/10.1007\%2F978-3-319-14240-1}{The Python Workbook}, 2015.
\item C. Johnson. \href{http://link.springer.com.ezproxy.uniandes.edu.co:8080/book/10.1007\%2F978-1-4302-1998-9}{Pro Bash Programming}, 2009.
\item I. Horton. \href{http://link.springer.com.ezproxy.uniandes.edu.co:8080/book/10.1007\%2F978-1-4302-0243-1}{Beginning C}, 2006.
\item B. Kernighan \& D. Ritchie. \textit{The C programming language.}
\item P. Scherer, \href{http://link.springer.com.ezproxy.uniandes.edu.co:8080/book/10.1007\%2F978-3-642-13990-1}{Computational Physics}, 2010.
\end{itemize}

%\item H. Langtangen. \href{http://link.springer.com.ezproxy.uniandes.edu.co:8080/book/10.1007\%2F978-3-540-73916-6}{Python Scripting for Computational Science}, 2008.
%\item\url{http://software-carpentry.org/}

\end{document}
