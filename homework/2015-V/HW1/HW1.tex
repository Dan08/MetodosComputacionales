
%--------------------------------------------------------------------
%--------------------------------------------------------------------
% Formato para los talleres del curso de Métodos Computacionales
% Universidad de los Andes
% 2015, curso de vacaciones
%--------------------------------------------------------------------
%--------------------------------------------------------------------

\documentclass[11pt,letterpaper]{exam}
\usepackage[utf8]{inputenc}
\usepackage[spanish]{babel}
\usepackage{graphicx}
\usepackage{mdframed}
\usepackage{tabularx}
\usepackage[absolute]{textpos} % Para poner una imagen en posiciones arbitrarias
\usepackage{multirow}
\mdfdefinestyle{mystyle}{leftmargin=1cm,rightmargin=1cm,linecolor=red}
\usepackage{float}
\usepackage{hyperref}
\decimalpoint

\newcommand{\base}[1]{\underline{\hspace{#1}}}
\boxedpoints
\pointname{ pt}
%\extrawidth{0.75in}
%\extrafootheight{-0.5in}
\extraheadheight{-0.15in}
\hypersetup{%
  colorlinks=true,% hyperlinks will be coloured
  urlcolor=blue
}

%\noprintanswers
%\printanswers
\renewcommand{\solutiontitle}{}
\SolutionEmphasis{\color{blue}}

\usepackage{upquote,textcomp}
\newcommand\upquote[1]{\textquotesingle#1\textquotesingle} % To fix straight quotes in verbatim

\begin{document}
\begin{center}
{\Large Métodos Computacionales} \\
Tarea 1 - \textsc{Unix}: línea de comandos \\
Mayo de 2015
\end{center}

\begin{textblock*}{40mm}(10mm,20mm)
  \includegraphics[width=3cm]{logoUniandes.png}
\end{textblock*}

\begin{textblock*}{40mm}(164mm,20mm)
  \includegraphics[width=3cm]{logoUniandes.png}
\end{textblock*}

\vspace{0.5cm}

La solución a este taller debe cargarse a su repositorio en GitHub en la carpeta /MC/Tareas/HW1/ y debe contener: \verb+arxiv.sh, bruno.sh, stardate.sh, lacita.txt+. Es requisito que en los scripts se pongan comentarios que expliquen lo que se está haciendo. La fecha límite para hacer un commit es el \textbf{miércoles 3 de junio a las 23:59}.

\vspace{0.5cm}

\begin{questions}
 
\question[15] {\bf{Una pequeña araña}}

El \href{http://en.wikipedia.org/wiki/ArXiv}{arXiv} es un repositorio de publicaciones científicas. Elija un tema y usando \verb+curl+, \verb+grep+, \verb+wc+, \verb+echo+ y \verb+sed+ escriba un script en \verb+bash+ llamado \verb+arxiv.sh+  que reciba una palabra clave y de regreso muestre la cantidad y el título de los nuevos artículos (\verb+http://arxiv.org/list/[el_tema]/new+) que contienen la palabra clave.

Por ejemplo, si se eligiera como tema la mecánica cuántica con la palabra clave \textbf{entanglement} el resultado al hacer \verb+./arxiv.sh entanglement+ debe ser similar al siguiente.

\begin{verbatim}
                __  ___       
        __ _ _ _\ \/ (_)_   __
       / _` | '__\  /| \ \ / /
      | (_| | |  /  \| |\ V / 
       \__,_|_| /_/\_\_| \_/  
                              
=====================================
Searching the arXiv for the new stuff
http://arxiv.org/list/quant-ph/new
=====================================
keyword: entanglement
=====================================
Articles found: 2
- Converting non-classicality into entanglement
- Area Law for Gapless States from Local Entanglement Thermodynamics
=====================================
\end{verbatim}

\question[25]{\bf{Planetas extrasolares}}

El archivo \href{https://raw.githubusercontent.com/ComputoCienciasUniandes/MetodosComputacionales/master/homework/2015-V/HW1/kepler.csv}{kepler.csv} tiene información astronómica sobre la mayoría de planetas extrasolares conocidos a la fecha con la especificación de las columnas en el archivo \href{https://raw.githubusercontent.com/ComputoCienciasUniandes/MetodosComputacionales/master/homework/2015-V/HW1/keplerREADME}{keplerREADME}. Escriba un script de \verb+bash+ llamado \verb+bruno.sh+ que haga lo siguiente.

\begin{parts}
	\part[5] Imprimir la cantidad de planetas incluidos en el catálogo. Usar \verb+awk+, \verb+wc+ y artimética con doble paréntesis.
	\part[10] Mostrar el nombre y la cantidad de planetas con una masa menor a una centésima de la masa de Júpiter. Usar \verb+awk+ y \verb+wc+.
	\part[10] Determinar el planeta con el menor periodo orbital. Usar \verb+sort+ con las siguientes opciones puede ser de utilidad \verb+sort --field-separator="," --key=6 -n+.
\end{parts}

\newpage

\question[30]{\bf{Historia estelar}}

El archivo \href{https://raw.githubusercontent.com/ComputoCienciasUniandes/MetodosComputacionales/master/homework/2015-V/HW1/hyg.csv}{hyg.csv} tiene información astronómica sobre las estrellas más brillantes en el firmamento. El archivo \href{https://raw.githubusercontent.com/ComputoCienciasUniandes/MetodosComputacionales/master/homework/2015-V/HW1/worldhistory.tsv}{worldhistory.tsv} tiene dos columnas: la primera el año y la segunda eventos históricos del año. Las columnas del archivo \verb+tsv+ están separadas por \verb+TAB+.

Escriba un script de \verb+bash+ llamado \verb+stardate.sh+ que reciba un año $x$ y de regreso muestre los eventos históricos del año junto con la orientación (RA y DEC) de estrellas (5 a lo sumo) cuya luz haya viajado entre $(2015-x)$ y $(2015-x)+1$ años y en consecuencia sea contemporánea con los eventos históricos mostrados.

\begin{verbatim}
./stardate.sh 1983
############################################
 ____  _             ____        _       
/ ___|| |_ __ _ _ __|  _ \  __ _| |_ ___ 
\___ \| __/ _` | '__| | | |/ _` | __/ _ \
 ___) | || (_| | |  | |_| | (_| | ||  __/
|____/ \__\__,_|_|  |____/ \__,_|\__\___|
                                         
############################################
 _  ___   ___ _____ 
/ |/ _ \ ( _ )___ / 
| | (_) |/ _ \ |_ \ 
| |\__, | (_) |__) |
|_|  /_/ \___/____/ 
                    
GPS becomes available for civilian use. Indep
endence of Brunei. End of dictatorship in Arg
entina. Second Sudanese Civil War begins. Inv
asion of Grenada by the United States. Bombin
g of U.S. Embassy in Beirut. The 1983 Beirut 
barracks bombing results in the deaths of 307
 people, hastening the removal of internation
 al peacekeeping forces in Lebanon.

############################################
LOOK AT THE FOLLOWING STARS:
RA/º    DEC/º   HIP No.
0.54    67.24   2552  
1.04    62.35   4872  
1.80    63.85   8362  
2.72    -54.12  12703 
4.63    52.89   21553 
############################################
\end{verbatim}
\newpage
\question[30] {\bf{Compufísica de noche}}

Cuando Compufísica esté cerrada (antes de las 7 AM o después de las 7 PM) lleve a cabo las siguientes tareas.
\begin{parts}
	\part[15] Haga \verb+ssh+ a compufi3 y ejecute la siguiente línea de código. \\
	\verb+echo "Hola Juan." | mail -s "Taller1" j-lizara+ \\
	\begin{flushright}{\bf{Tiene}} que ser {\bf{compufi3}}.\end{flushright}
	\part[15] Inicie una sesión \verb+sftp+ a alguna máquina de Compufísica, ingrese a la carpeta \\ \verb+/usuarios/homenfs7/taller1/+ y descargue con \verb+get+ el archivo con su nombre a su computador, abra la imagen y copie el texto a un archivo llamado \verb+lacita.txt+. 
\end{parts}
\end{questions}

\end{document}
