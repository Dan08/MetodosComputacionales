
%--------------------------------------------------------------------
%--------------------------------------------------------------------
% Formato para los talleres del curso de Métodos Computacionales
% Universidad de los Andes
% 2015, curso de vacaciones
%--------------------------------------------------------------------
%--------------------------------------------------------------------

\documentclass[11pt,letterpaper]{exam}
\usepackage[utf8]{inputenc}
\usepackage[spanish]{babel}
\usepackage{graphicx}
\usepackage{mdframed}
\usepackage{tabularx}
\usepackage[absolute]{textpos} % Para poner una imagen en posiciones arbitrarias
\usepackage{multirow}
\mdfdefinestyle{mystyle}{leftmargin=1cm,rightmargin=1cm,linecolor=red}
\usepackage{float}
\usepackage{hyperref}
\decimalpoint

\newcommand{\base}[1]{\underline{\hspace{#1}}}
\boxedpoints
\pointname{ pt}
%\extrawidth{0.75in}
%\extrafootheight{-0.5in}
\extraheadheight{-0.15in}
\hypersetup{%
  colorlinks=true,% hyperlinks will be coloured
  urlcolor=blue
}

%\noprintanswers
%\printanswers
\renewcommand{\solutiontitle}{}
\SolutionEmphasis{\color{blue}}

\usepackage{upquote,textcomp}
\newcommand\upquote[1]{\textquotesingle#1\textquotesingle} % To fix straight quotes in verbatim

\begin{document}
\begin{center}
{\Large Métodos Computacionales} \\
Taller 1 - \textsc{Unix}: línea de comandos \\
Mayo de 2015
\end{center}

\begin{textblock*}{40mm}(10mm,20mm)
  \includegraphics[width=3cm]{logoUniandes.png}
\end{textblock*}

\begin{textblock*}{40mm}(161mm,20mm)
  \includegraphics[width=3cm]{logoUniandes.png}
\end{textblock*}

\vspace{0.5cm}

Los 3 scripts de solución de este taller deben ser presentados en un solo archivo con nombre \verb+NombreApellido_HW1.zip+ en \textbf{sicuaplus}. Por ejemplo: yo deber\'ia subir un archivo llamado \verb"SebastianPerez_HW1.zip"

En cada parte del ejercicio se entrega 1/3  de los puntos si el código propuesto es razonable, 1/3 si se puede ejecutar y 1/3 si entrega resultados correctos.

\vspace{0.5cm}

\begin{questions}
 
\question[45] {\bf{Un pequeño robot}}

El \href{http://en.wikipedia.org/wiki/ArXiv}{arXiv} es un repositorio de publicaciones científicas. Elija un tema y usando \verb+curl+, \verb+grep+, \verb+wc+, \verb+echo+ y \verb+sed+ escriba un script llamado arxiv.sh en \verb+bash+ que reciba una palabra clave y de regreso muestre la cantidad y el título de los nuevos artículos (\verb+http://arxiv.org/list/tema/new+) que contienen la palabra clave.

Por ejemplo, si se eligiera el tema de mecánica cuántica con la palabra clave \textbf{entanglement} el resultado al hacer \verb+./arxiv.sh entanglement+ debe ser similar al siguiente.

\begin{verbatim}
                __  ___       
        __ _ _ _\ \/ (_)_   __
       / _` | '__\  /| \ \ / /
      | (_| | |  /  \| |\ V / 
       \__,_|_| /_/\_\_| \_/  
                              
=====================================
Searching the arXiv for the new stuff
http://arxiv.org/list/quant-ph/new
=====================================
keyword: entanglement
=====================================
Articles found: 2
- Converting non-classicality into entanglement
- Area Law for Gapless States from Local Entanglement Thermodynamics
=====================================
\end{verbatim}

\question[10]{\bf{Planetas extrasolares}}

El archivo \href{keplerurl}{kepler.csv} tiene información astronómica sobre la mayoría de planetas extrasolares conocidos a la fecha con la especificación de las columnas en el archivo \href{keplerurl}{keplerREADME}. Escriba un script de \verb+bash+ llamado \verb+bruno.sh+ que haga lo siguiente.

\begin{parts}
	\part Imprimir la cantidad de planetas incluidos en el catálogo.
	\part Mostrar el nombre y la cantidad de planetas con una masa menor a una centésima de la masa de Júpiter.
	\part Determinar el planeta con el menor periodo orbital.
\end{parts}

\question[10]{\bf{Historia estelar}}

El archivo \href{http://www.google.com}{hyg.csv} tiene información astronómica sobre las estrellas más brillantes en el firmamento. El archivo \href{http://www.google.com}{worldhistory.tsv} tiene dos columnas: la primera el año y la segunda eventos históricos.

\begin{parts}
	\part Escriba un script de \verb+bash+ llamado \verb+stardate.sh+ que reciba un año y de regreso muestre los eventos históricos del año junto con las estrellas cuya luz ha viajado tantos años como el tiempo transcurrido desde el año entregado hasta el 2015.
\end{parts}

\begin{verbatim}
./stardate.sh 1983
############################################
 ____  _             ____        _       
/ ___|| |_ __ _ _ __|  _ \  __ _| |_ ___ 
\___ \| __/ _` | '__| | | |/ _` | __/ _ \
 ___) | || (_| | |  | |_| | (_| | ||  __/
|____/ \__\__,_|_|  |____/ \__,_|\__\___|
                                         
############################################
 _  ___   ___ _____ 
/ |/ _ \ ( _ )___ / 
| | (_) |/ _ \ |_ \ 
| |\__, | (_) |__) |
|_|  /_/ \___/____/ 
                    
GPS becomes available for civilian use. Indep
endence of Brunei. End of dictatorship in Arg
entina. Second Sudanese Civil War begins. Inv
asion of Grenada by the United States. Bombin
g of U.S. Embassy in Beirut. The 1983 Beirut 
barracks bombing results in the deaths of 307
 people, hastening the removal of internation
 al peacekeeping forces in Lebanon.

############################################
LOOK AT THE FOLLOWING STARS:
RA/º    DEC/º   HIP No.
0.54    67.24   2552  
1.04    62.35   4872  
1.80    63.85   8362  
2.72    -54.12  12703 
4.63    52.89   21553 
############################################
\end{verbatim}

\end{questions}

\end{document}
