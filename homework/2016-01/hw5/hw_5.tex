\documentclass{article}
\textheight=25.5cm
\textwidth=16.0cm
\oddsidemargin=-0.5cm
\topmargin=-3.0cm
\usepackage[pdftex]{graphicx}
\usepackage[utf8]{inputenc}
\usepackage[spanish]{babel}
\title{Taller \#5 de M\'etodos Computacionales\\ FISI 2028, Semestre 2016 - 10}
\author{Profesor: Germán Chaparro}
\date{Jueves 7 de Abril, 2016}
\begin{document}
\maketitle
\thispagestyle{empty}


{\bf Importante}
\begin{itemize}

\item Los siete archivos con el c\'odigo fuente que soluciona esta
  tarea deben subirse a trav\'es de sicuaplus antes de las media noche del
  martes 12 de Abril como un \'unico archivo zip con el nombre
  \verb"NombreApellidos_hw5.zip", por ejemplo yo deber\'ia subir un
  archivo llamado \verb"GermanChaparro_hw5.zip"

\item La nota m\'axima de este taller es de 100 puntos. Los puntos indicados
en cada literal solamente se otorgan si el programa compila y da los
resultados esperados seg\'un la descripci\'on de cada punto.
 

\item El archivo del genoma de la Vibrio cholerae se encuentra en este
  repositorio:

  \verb"https://github.com/forero/ComputationalMethodsData/tree/master/homework/hw_2"
\end{itemize}


\begin{enumerate}

\item (35 puntos) Resuelva todos los ejercicios de los capítulos 1 y 2 dellibro de Horton, guarde las respuestas en los archivos: \verb+horton1-1.c+, \verb+horton1-2.c+, \verb+horton1-3.c+, \verb+horton2-1.c+, \verb+horton2-2.c+, \verb+horton2-3.c+, \verb+horton2-4.c+.

\item
\begin{itemize}
\item[a)] (10 puntos) Escriba un programa en C que genere un archivo
  con $n$ filas y $m$ columnas de n\'umeros aleatorios entre $0$ y
  $1$. El c\'odigo fuente debe estar en un archivo llamado
  \verb"gen_random.c". El programa debe poder ejecutarse como
  \verb"gen_random.x n m filename", donde \verb"filename" es un nombre
  arbitrario del archivo donde se van a escribir los datos. 

\item[b)] (10 puntos) 
Escriba un programa en C que lea un archivo de nombre arbitrario que
contiene $n$ filas y $m$ columnas de n\'umeros escritos en el mismo
formato que usa \verb"gen_random.c" para escribir los datos. El
c\'odigo fuente debe estar en un archivo llamado
\verb"max_random.c". El c\'odigo debe adem\'as imprimir en pantalla el
valor m\'aximo en cada una de las columnas. El programa debe poder
ejecutarse como \verb"max_random.x n m filename", donde
\verb"filename" es un nombre arbitrario del archivo que contiene los
datos. 

\end{itemize}

\item
\begin{itemize}
\item[a)] (10 puntos) El archivo \verb"Vibrio_cholerae.txt" contiene
  el genoma de la bacteria Vibrio cholerae. Escriba un programa en C
  llamado \verb"patron.c" que encuentre los {\bf dos} patrones de 5 bases
  consecutivas que m\'as se encuentren en las primeras $10^4$ bases
  del genoma de la Vibrio cholerae. 

\item[b)] (25 puntos) Escriba un programa en C llamado
  \verb"patron_nm.c" que encuentre los dos patrones de $n$ bases
  consecutivas que m\'as se encuentren en las primeras $m$ bases del
  genoma de la Vibrio cholerae. Este programa debe poder ejecutarse
  desde consola como \verb"patron_nm.x n m". El programa debe
  verificar que $n>0$, $m>0$, $n\leq m$ y que $m$ es menor que el
  n\'umero de bases en el genoma. Si alguna de esas condiciones no se
  cumple, el programa debe escribir un mensaje explicando el problema
  antes de parar su ejecuci\'on. 
\end{itemize}



\end{enumerate}

\end{document}
