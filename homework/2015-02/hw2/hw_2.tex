
%--------------------------------------------------------------------
%--------------------------------------------------------------------
% Formato para los talleres del curso de Métodos Computacionales
% Universidad de los Andes
% 2015-10
%--------------------------------------------------------------------
%--------------------------------------------------------------------

\documentclass[11pt,letterpaper]{exam}
\usepackage[utf8]{inputenc}
\usepackage[spanish]{babel}
\usepackage{graphicx}
\usepackage{tabularx}
\usepackage[absolute]{textpos} % Para poner una imagen completa en la portada
\usepackage{multirow}
\usepackage{float}
\usepackage{hyperref}
\decimalpoint
%\usepackage{pst-barcode}
%\usepackage{auto-pst-pdf}

\newcommand{\base}[1]{\underline{\hspace{#1}}}
\boxedpoints
\pointname{ pt}
%\extrawidth{0.75in}
%\extrafootheight{-0.5in}
\extraheadheight{-0.15in}
%\pagestyle{head}

%\noprintanswers
%\printanswers


\usepackage{upquote,textcomp}
\newcommand\upquote[1]{\textquotesingle#1\textquotesingle} % To fix straight quotes in verbatim

\begin{document}
\begin{center}
{\Large Métodos Computacionales} \\
Taller 2 \\
Profesor: Jaime Forero\\
Fecha de Publicación: {\small \it Agosto 18 de 2015}\\
\end{center}

\begin{textblock*}{40mm}(10mm,20mm)
  \includegraphics[width=3cm]{logoUniandes.png}
\end{textblock*}

\begin{textblock*}{40mm}(161mm,20mm)
  \includegraphics[width=3cm]{logoUniandes.png}
\end{textblock*}

\vspace{0.5cm}

{\Large Instrucciones de Entrega}\\

\noindent
La solución a este taller debe subirse por SICUA antes de las 8:30AM
del jueves 27 de Agosto del 2015. 
\noindent
Si la soluci\'on est\'a en SICUA
antes de las 8:30AM del domingo 23 de Agosto del 2015 se calificar\'a
el taller sobre 130 puntos. 
\noindent
Los tres archivos c\'odigo fuente deben subirse en un \'unico archivo
\verb".zip" con el nombre \verb"NombreApellido_hw2.zip", por ejemplo
yo deber\'ia subir el zip \verb"JaimeForero_hw2.zip".
\noindent
Todas los algoritmos deben ser implementados con funciones b\'asicas
de python. Librer\'ias externas solamente son permitidas al momento de
escribir/leer datos o hacer gr\'aficas.


\begin{questions}

\question[30 (40)] {\bf{Barajando Arreglos}} En este punto van a
implementar un algoritmo para desordenar arreglos. 

\begin{parts}
\part[20 (25)] Escriba una funci\'on que desordene un arreglo de \emph{n}
elementos de manera aleatoria (puede pensar en una baraja de cartas,
por ejemplo). La funci\'on debe poder utilizarse desde python de la
siguiente manera:
\begin{verbatim}
import barajando as baraja
import numpy as np
a = np.arange(10) #este es un array inicial de 10 elementos
b = baraja.baraja(a) #b es el array barajado
\end{verbatim}
\part[10 (15)] Escriba una funci\'on para visualizar la efectividad de
este algoritmo. La funci\'on debe generar una matriz $A_{ij}$ donde el
elemento $ij$ corresponde al n\'umero de veces que el elemento que
inicialmente se encuentra en la posici\'on $i$ termina en la
posici\'on $j$ si se hicieran muchas iteraciones desordenando el array.  La
funci\'on debe poder utilizarse desde python de la siguiente manera.

\begin{verbatim}
import barajando as baraja
matriz_a = baraja.matriz(baraja.baraja, n_lado_matriz=100, n_iter=10000)
\end{verbatim}
\end{parts}


\question[30 (40)] {\bf{Volando sobre una esfera}}

Los aviones que vuelan largas distancias internacionales buscan
minimizar su consumo de combustible. Para esto la ruta de vuelo debe
ser la m\'as corta entre los puntos de partida y llegada.  El objetivo
de este ejercicio es escribir un programa que calcule las coordenadas
de esa trayectoria para dos puntos arbitrarios sobre la Tierra. 

Para simplificar el problema supondremos que la Tierra es
perfectamente esf\'erica.  Adem\'as vamos a utilizar los \'angulos
$\theta$ (polar) y $\varphi$ (ecuatorial) para describir las
coordenadas sobre un punto de la esfera .

Escriba una funci\'on que dados dos puntos arbitrarios sobre la
esfera (parametrizados por $\theta$ y $\varphi$) calcula la
trayectoria m\'as corta que entre esos dos puntos. La funci\'on debe
poder utilizarse de la siguiente manera 


\begin{verbatim}
import esfera as sf
theta_inicial = 0 # en grados
phi_inicial = 30 # en grados
theta_final  = 240
phi_final = 10 
n_points = 1000 # numero de puntos para la trayectoria
camino_theta, camino_phi = sf.camino(theta_inicial, phi_inicial,
theta_final phi_final, n_points)
\end{verbatim}

\question[40 (50)] {\bf{Cambio de coordenadas}}
Como vemos cada noche las estrellas realizan un movimiento aparente en
el cielo, en realidad este movimiento se debe a la rotación de la
tierra sobre su propio eje y alrededor del Sol. En el catalogo
\verb+hipparcos.csv+ \footnote{\url{https://github.com/ComputoCienciasUniandes/HerramientasComputacionalesDatos/tree/master/data/Hipparcos}} 
se encuentra la siguiente información  \verb+id AR DEC Mag Distance+
de cada estrella, Donde  \verb+AR+ y \verb+DEC+ son las coordenadas
ecuatoriales \footnote{\url{http://es.wikipedia.org/wiki/Coordenadas_ecuatoriales}}
de las estrellas. \verb+Mag+ Esta relacionado con la luminosidad de la
estrella y \verb+Distance+ es la distancia a la estrella en
parsecs \footnote{\url{http://es.wikipedia.org/wiki/Parsec}}. 

Escriba una funci\'on que transforme las
coordendas del sistema ecuatorial (\verb+AR+ y \verb+DEC+) a las
coordenadas horizontales de \verb+Altura+ y \verb+Azimut+. La
funci\'on se debe poder utilizar sobre los datos de Hipparcos y
llamarse desde python de la siguiente manera 
\begin{verbatim}
import coordenadas as coor
filename = 'hipparcos.csv'
latitud = +45.0 # en grados
longitud = -34.0 # en grados
mes = 10
dia = 14
anno = 2014
hora = 20
minutos = 10
segundos = 30
hipp_ra, hipp_dec = coor.load_data(filename)
hipp_alt, hipp_az = coor.ec2hor(hipp_ra, hipp_dec, 
lat=latitud, long=longitud, y=anno, m=mes, d=dia, h=hora, min=minutos,
s=segundos)  
\end{verbatim}



\end{questions}
\end{document}
