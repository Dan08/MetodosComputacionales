
%--------------------------------------------------------------------
%--------------------------------------------------------------------
% Formato para los talleres del curso de Métodos Computacionales
% Universidad de los Andes
% 2015-10
%--------------------------------------------------------------------
%--------------------------------------------------------------------

\documentclass[11pt,letterpaper]{exam}
\usepackage[utf8]{inputenc}
\usepackage[spanish]{babel}
\usepackage{graphicx}
\usepackage{tabularx}
\usepackage[absolute]{textpos} % Para poner una imagen completa en la portada
\usepackage{multirow}
\usepackage{float}
\usepackage{hyperref}
\decimalpoint
%\usepackage{pst-barcode}
%\usepackage{auto-pst-pdf}

\newcommand{\base}[1]{\underline{\hspace{#1}}}
\boxedpoints
\pointname{ pt}
%\extrawidth{0.75in}
%\extrafootheight{-0.5in}
\extraheadheight{-0.15in}
%\pagestyle{head}

%\noprintanswers
%\printanswers


\usepackage{upquote,textcomp}
\newcommand\upquote[1]{\textquotesingle#1\textquotesingle} % To fix straight quotes in verbatim

\begin{document}
\begin{center}
{\Large Métodos Computacionales} \\
Taller 3 \\
Profesor: Jaime Forero\\
Fecha de Publicación: {\small \it Septiembre 1 de 2015}\\
\end{center}

\begin{textblock*}{40mm}(10mm,20mm)
  \includegraphics[width=3cm]{logoUniandes.png}
\end{textblock*}

\begin{textblock*}{40mm}(161mm,20mm)
  \includegraphics[width=3cm]{logoUniandes.png}
\end{textblock*}

\vspace{0.5cm}

{\Large Instrucciones de Entrega}\\

\noindent
La solución a este taller debe subirse por SICUA antes de las 8:30AM
del jueves 11 de Septiembre del 2015. 
\noindent
Si la soluci\'on est\'a en SICUA
antes de las 8:30AM del domingo 6 de Septiembre 2015 se calificar\'a
el taller sobre 130 puntos. 
\noindent
Los tres archivos c\'odigo fuente deben subirse en un \'unico archivo
\verb".zip" con el nombre \verb"NombreApellido_hw2.zip", por ejemplo
yo deber\'ia subir el zip \verb"JaimeForero_hw2.zip".
\noindent
Todas los algoritmos deben ser implementados con funciones b\'asicas
de python. Librer\'ias externas solamente son permitidas al momento de
escribir/leer datos o hacer gr\'aficas.

\begin{questions}

\question[25 (35)] {\bf{Patrones Ocultos}} El siguiente archivo
\verb"hands_on/notas_fisicaII_201410.dat" del repositorio
\verb"MetodosComputacionalesDatos" contiene datos de notas finales de
un curso de F\'isica II. Las tres primeras columnas corresponden a
parciales, la siguiente a la nota en complementaria y la quinta al
examen final. Haga un an\'alisis de PCA sobre esas cinco columnas de
tal manera que se puedan imprimir de manera ordenada los autovalores y
autovectores. El an\'alisis solamente debe tener en cuenta a las
personas que presentaron todos los parciales.

\begin{verbatim}
import notas
valores, vectores = notas.pca("notas_fisica_II_201410.dat")
for val, vec in zip(valores, vectores):
    print val, vec
\end{verbatim}


\question[25 (35)] {\bf{Ciclo Solar}}
El siguiente archivo \verb"hands_on/solar/monthrg.dat" tiene datos del
n\'umero de manchas solares cada mes desde 1610. La primera columna
corresponde al a\~no de la medici\'on, la segunda al mes, la tercera
al n\'umero de d\'ias que se pudo observar el n\'umero de manchas y la
cuarta al n\'umero promedio de manchas. Si hay un \verb"-99" en la
cuarta columna quiere decir que el n\'umero de manchas promedio se
desconoce para ese mes.

Asumiendo que este ciclo puede describirse por la siguiente funci\'on:
\begin{displaymath}
f(t) = a\cos((2\pi/11) t + b) + c
\end{displaymath}
donde $t$ es el tiempo en a\~nos, escriba una funci\'on que tome los
datos y haga un ajuste por m\'inimos cuadrados a los par\'ametros
libres de la siguiente manera

\begin{verbatim}
import numpy as np.
import solar
[a,b,c] = solar.ajuste("monthrg.dat", min=1900, max=1980)
print a, b, c
\end{verbatim}

\question[50 (60)] {\bf{Caras en PCA}}
Como aplicaci\'on del m\'etodo de PCA van a implementar rutinas
sencillas para programas de reconocimiento facial.

Vamos a tomar como punto de partida el conjunto de la base de datos de
ATT
\url{http://www.cl.cam.ac.uk/research/dtg/attarchive/facedatabase.html}. 

Implemente una funci\'on que encuentre los autovalores y autovectores
correspondientes a las caras en la base de datos y grafica las
primeras $n$ (ordenadas en orden decreciente de importancia seg\'un
los autovalores) en un archivo pdf. 

La funci\'on debe funcionar de la siguiente manera
\begin{verbatim}
import caras
[valores, vectores] = caras.pca(pathin="/home/forero/data/at")
n_faces = 10
caras.make_plot(valores, vectores, n=n_faces, fileout="eigenfaces.pdf")
\end{verbatim}

\end{questions}
\end{document}
