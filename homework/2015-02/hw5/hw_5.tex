%--------------------------------------------------------------------
%--------------------------------------------------------------------
% Formato para los talleres del curso de Métodos Computacionales
% Universidad de los Andes
% 2015-20
%--------------------------------------------------------------------
%--------------------------------------------------------------------

\documentclass[11pt,letterpaper]{exam}
\usepackage[utf8]{inputenc}
%\usepackage[spanish]{babel}
\usepackage{graphicx}
\usepackage{tabularx}
\usepackage[absolute]{textpos} % Para poner una imagen completa en la portada
\usepackage{hyperref}

\newcommand{\base}[1]{\underline{\hspace{#1}}}
\boxedpoints
\pointname{ pt}

\extraheadheight{-0.15in}

\newcommand\upquote[1]{\textquotesingle#1\textquotesingle} % To fix straight quotes in verbatim

\begin{document}
\begin{center}
{\Large Métodos Computacionales} \\
Taller 4 \\
Profesor: Jaime Forero\\
Fecha de Publicación: {\small \it Octubre 6 de 2015}\\
\end{center}

\begin{textblock*}{40mm}(10mm,20mm)
  \includegraphics[width=3cm]{logoUniandes.png}
\end{textblock*}

\begin{textblock*}{40mm}(161mm,20mm)
  \includegraphics[width=3cm]{logoUniandes.png}
\end{textblock*}

\vspace{0.5cm}

{\Large Instrucciones de Entrega}\\

\noindent
La solución a este taller debe subirse por SICUA antes de las 8:30AM
del jueves 15 de Octubre del 2015. 
\noindent
Si la soluci\'on est\'a en SICUA
antes de las 8:30AM del domingo 11 de Octubre 2015 se calificar\'a
el taller sobre 120 puntos. 
\noindent
Esta tarea solamente vale el $10\%$ de la nota final (todas las otras
tareas valen el $15\%$).
\noindent
Cada punto debe tener la respuesta en un c\'odigo fuente de C
separado. Los c\'odigos deben subirse en un solo archivo
\verb".zip" con el nombre \verb"NombreApellido_hw5.zip", por ejemplo
yo deber\'ia subir el zip \verb"JaimeForero_hw5.zip".

\noindent
Los archivos de datos necesarios se encuentran aqu\'i:

\noindent
\url{https://github.com/ComputoCienciasUniandes/MetodosComputacionalesDatos/tree/master/homework/2014-20/hw_5} 
\begin{questions}



\question[50 (60)] {\bf{Precisi\'on de integraci\'on montecarlo}} 

Implemente un m\'etodo montecarlo para calcular la siguiente integral
\begin{displaymath}
\int_{-5}^5 e^{-x^2}\mathrm{d}x.
\end{displaymath}

Escriba un programa en $C$ (\verb"montecarlo.c") que imprime en
pantalla el valor de la integral y el n\'umero de puntos montecarlo a
partir del cual la integral ha convergido por debajo de un valor $h$. 

El programa debe poder ejecutarse como sigue

\begin{verbatim}
./montecarlo h
\end{verbatim}
donde \verb"h" es el valor de $h$ correspondiente.



\question[50 (60)] {\bf{La funci\'on gamma}}

Una funci\'on que aparece comunmente en c\'alculos f\'isicos es la
funci\'on gama la cual est\'a definida por la siguiente integral.
\begin{equation}
\Gamma(a) = \int_{0}^{\infty} x^{a-1}e^{-x}\mathrm{d}x
\end{equation}

En primer lugar inspeccionen la forma del integrando haciendo
gr\'aficas para $a=2,3,4$ en el intervalo $0<x<5$. Verifiquen
anal\'iticamente que el m\'aximo se encuentra en $x=a-1$. Esto quiere
decir que lo importante para la integral va a ser capturar bien la
funci\'on alrededor de este m\'aximo.

Para calcular esta integral indefinida ahora van a hacer el siguiente
cambio de variable

\begin{equation}
z = \frac{x}{c+x}
\end{equation}
donde $c$ es una constante tal que el m\'aximo de la funci\'on ahora
se encuentra en $z=1/2$.

De esta manera ahora los l\'imites de la nueva integral son definidos.
Otro cambio que deben tener en cuenta para hacer la integral tratable
num\'ericamente es reescribir el t\'ermino $x^{a-1}$ como $e^{{a-1}\ln
  x}$.

Ahora s\'i, escriba un programa (\verb"factorial.c") en $C$ que
imprima en pantalla el valor de esta integral para cualquier $a>0$. El
ejecutable debe poder llamarse como 
\begin{verbatim}
./factorial a
\end{verbatim}
donde \verb"a" es el valor de la variable $a$. 

Recuerde que para valores enteros de $a$, $\Gamma(a)=(a-1)!$


\end{questions}
\end{document}
