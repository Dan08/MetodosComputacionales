%--------------------------------------------------------------------
%--------------------------------------------------------------------
% Formato para los talleres del curso de Métodos Computacionales
% Universidad de los Andes
% 2015-20
%--------------------------------------------------------------------
%--------------------------------------------------------------------

\documentclass[11pt,letterpaper]{exam}
\usepackage[utf8]{inputenc}
%\usepackage[spanish]{babel}
\usepackage{graphicx}
\usepackage{tabularx}
\usepackage[absolute]{textpos} % Para poner una imagen completa en la portada
\usepackage{hyperref}

\newcommand{\base}[1]{\underline{\hspace{#1}}}
\boxedpoints
\pointname{ pt}

\extraheadheight{-0.15in}

\newcommand\upquote[1]{\textquotesingle#1\textquotesingle} % To fix straight quotes in verbatim

\begin{document}
\begin{center}
{\Large Métodos Computacionales} \\
Taller 6 \\
Profesor: Jaime Forero\\
Fecha de Publicación: {\small \it Octubre 26 de 2015}\\
\end{center}

\begin{textblock*}{40mm}(10mm,20mm)
  \includegraphics[width=3cm]{logoUniandes.png}
\end{textblock*}

\begin{textblock*}{40mm}(161mm,20mm)
  \includegraphics[width=3cm]{logoUniandes.png}
\end{textblock*}

\vspace{0.5cm}

{\Large Instrucciones de Entrega}\\

\noindent
La solución a este taller debe subirse por SICUA antes de las 10:00AM
del jueves 5 de Noviembre del 2015. 
\noindent
Si la soluci\'on est\'a en SICUA
antes de las 8:30AM del s\'abado 31 de Octubre del 2015 se calificar\'a
el taller sobre 140 puntos. 
\noindent
Los c\'odigos deben estar en un \'unico repositorio de github con un
\'ultimo commit hecho antes de la fecha l\'imite de entrega. El
repositorio debe tener dos carpetas de nombre \verb"magnetico" y
\verb"burgers" para cada uno de los ejercicios A SICUA
solamente se debe responder con la direcci\'on del repositorio.


\begin{questions}
\question[50 (70)] {\bf{Movimiento de una part\'icula cargada en el
   el campo magn\'etico de la Tierra}}


La Tierra tiene un campo magn\'etico que la protege de part\'iculas
cargadas. Por ejemplo, la aurora boreal es un fen\'omeno natural que se puede
entender a partir del moviemiento de part\'iculas cargadas que se
aceleran dentro de un campo magn\'etico y empizan a radiar energ\'ia
electromagn\'etica.  

El campo magn\'etico de la Tierra, cerca de ella y hasta una distancia
de 4$R_{T}$ donde $R_T=6378.1$km es el radio de  la Tierra, puede ser
aproximado por el campo de un dipolo 

\begin{equation}
\vec{B}_{dip}(\vec{r}) = \frac{\mu_0}{4\pi r^3}[3(\vec{M}\cdot\vec{
    r})\hat{r}-\vec{M}], 
\end{equation}

donde $\vec{r}=x\hat{x} + y\hat{y} + z\hat{z}$, $r=|\hat{r}|$ y
$\hat{r}=\vec{r}/r$. Para la Tierra tomamos ${\bf M}=-M\hat{z}$
  antiparalelo al eje $z$ porque el polo norte magn\'etico es cercano
  al polo sur geogr\'afico. 

En el ecuador magn\'etico $(x=1R_T,y=z=0)$ la intensidad del campo es
$B_0=3\times 10^{-5}$T. Substitutendo esto en la ecuaci\'on anterior
tenermos que $\mu M/4\pi=B_0R_T^3$. Con esto podemos escribir en
coordenadas cartesianas:

\begin{equation}
\vec{B}_{dip} = -\frac{B_0 R_T^3}{r^5}[3xz\hat{x} + 3yz\hat{y} + (2z^2
  - x^2 -y^2)\hat{z}].
\end{equation}

En este problema, las velocidades de las part\'iculas est\'an en el
r\'egimen relativista. En este caso se tiene que la ecuaci\'on de la
fuerza se escribe como  
 
\begin{equation}
\frac{d(\gamma m \vec{v})}{dt} = q\vec{v}\times\vec{B}, 
\end{equation}
%
donde $\gamma=1/\sqrt{1-v^2/c^2}$. En este caso la energ\'ia
cin\'etica de la part\'icula se escribe como $K=(\gamma -1)mc^2$,
donde $m$ es la masa en reposo de la part\'icula.  

\begin{itemize}

\item (30 (40) puntos) Escriba un programa en C que integre las ecuaciones
  de movimiento relativista de {\bf protones} que tienen una
  posici\'on inicial $(2R_e,0,0)$, energ\'ia cin\'etica inicial
  $E_{k}$ (medida en millones de electronvolts) y tienen una velocidad
  inicial descrita por $(0,v\sin\alpha, v\cos\alpha)$ donde $\alpha$
  es un \'angulo que se conoce como el pitch angle. El programa debe
  poder ejecutarse de la siguiente manera 
\begin{verbatim}
./particle_in_field.x kinetic_energy alpha
\end{verbatim}
donde \verb"kinetic_energy" es la energ\'ia cin\'etica inicial medida
en megaelectronvolts y \verb"alpha" es el \'angulo $\alpha$ medido en
grados. La trayectoria debe seguirse por $100$ segundos y al final
debe escribir el tiempo y las coordenadas $x,y,z$ de las posiciones en un archivo
llamado \verb"trayectoria_E_alpha.dat" donde \verb"E" es el valor de la
energ\'ia cin\'etica de la part\'icula y \verb"alpha" es el valor del
\'angulo $\alpha$.

\item (10 (15) puntos) Escriba un programa en Python que grafique en el
  plano {\it xy}, y en {\it xyz} la trayectoria guardada en
  \verb"trayectoria_E_alpha.dat". La gr\'afica se debe guardar como un
  archivo pdf.

\item (10 (15) puntos) Escriba un \verb"makefile" que enlace correctamente
  todos los pasos anteriores.
\end{itemize}




\question[50 (70)] {\bf{Ecuaci\'on de Burgers en 2D}}


\begin{itemize}
\item (30(40) puntos) Escriba un programa en C que resuelva la
  ecuaci\'on de Burgers usando las mismas aproximaciones y variables
  mostradas en el trabajo de Lorena Barba para 500 pasos de tiempo:\\
\url{http://nbviewer.ipython.org/github/barbagroup/CFDPython/blob/master/lessons/10_Step_8.ipynb}

\item  (10 (15) puntos) Escriba un program en Python que genere un gif
  animado con la evoluci\'on temporal de la soluci\'on.

\item (10 (15) puntos) Escriba un \verb"makefile" que enlace
  correctamente todos los pasos anteriores.
\end{itemize}


\end{questions}
\end{document}
