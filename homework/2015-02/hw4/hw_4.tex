
%--------------------------------------------------------------------
%--------------------------------------------------------------------
% Formato para los talleres del curso de Métodos Computacionales
% Universidad de los Andes
% 2015-10
%--------------------------------------------------------------------
%--------------------------------------------------------------------

\documentclass[11pt,letterpaper]{exam}
\usepackage[utf8]{inputenc}
\usepackage[spanish]{babel}
\usepackage{graphicx}
\usepackage{tabularx}
\usepackage[absolute]{textpos} % Para poner una imagen completa en la portada
\usepackage{multirow}
\usepackage{float}
\usepackage{hyperref}
\decimalpoint
%\usepackage{pst-barcode}
%\usepackage{auto-pst-pdf}

\newcommand{\base}[1]{\underline{\hspace{#1}}}
\boxedpoints
\pointname{ pt}
%\extrawidth{0.75in}
%\extrafootheight{-0.5in}
\extraheadheight{-0.15in}
%\pagestyle{head}

%\noprintanswers
%\printanswers


\usepackage{upquote,textcomp}
\newcommand\upquote[1]{\textquotesingle#1\textquotesingle} % To fix straight quotes in verbatim

\begin{document}
\begin{center}
{\Large Métodos Computacionales} \\
Taller 4 \\
Profesor: Jaime Forero\\
Fecha de Publicación: {\small \it Septiembre 15 de 2015}\\
\end{center}

\begin{textblock*}{40mm}(10mm,20mm)
  \includegraphics[width=3cm]{logoUniandes.png}
\end{textblock*}

\begin{textblock*}{40mm}(161mm,20mm)
  \includegraphics[width=3cm]{logoUniandes.png}
\end{textblock*}

\vspace{0.5cm}

{\Large Instrucciones de Entrega}\\

\noindent
La solución a este taller debe subirse por SICUA antes de las 8:30AM
del jueves 1 de Octubre del 2015. 
\noindent
Si la soluci\'on est\'a en SICUA
antes de las 8:30AM del domingo 20 de Septiembre 2015 se calificar\'a
el taller sobre 130 puntos. 
\noindent
Cada punto debe tener la respuesta en un notebook de IPython/Jupyter por
separado. Los notebooks deben subirse en un \'unico archivo
\verb".zip" con el nombre \verb"NombreApellido_hw4.zip", por ejemplo
yo deber\'ia subir el zip \verb"JaimeForero_hw4.zip".

\noindent
Los archivos de datos necesarios se encuentran aqu\'i:

\noindent
\url{https://github.com/ComputoCienciasUniandes/MetodosComputacionalesDatos/tree/master/homework/2014-20/hw_5} 
\begin{questions}



\question[30 (40)] {\bf{Limpiando ruido (\verb"limpiando.ipynb")}} 


En clase trabajamos el filtrado de una se\~nal unidimensional
quit\'andole las frecuencias altas y dejando las frecuencias
bajas. Ahora vamos a intentar algo similar con una imagen, es decir con
una se\~nal bidimensional. Vamos a intentar dos cosas
diferentes: en un caso dejar las frecuencias altas y en otro caso
dejar las frecuencias bajas.

Escriba un programa en Python que haga el filtrado de una imagen de dos maneras. La primera que deje pasar las frecuencias bajas; la segunda que deje pasar las frecuencias altas. 

En ambos casos implemente un filtro suave. Ver la siguiente
referencia: \url{http://paulbourke.net/miscellaneous/imagefilter/}. 

Muestre los resultados de aplicar los filtros para las siguientes
im\'agenes 
\begin{itemize}
\item \verb"full_moon.jpg"
\item \verb"20_popc_cho009-1.tif"
\item \verb"colesterol-1.tif" 
\item \verb"ves_full_150_002-1.tif"
\end{itemize}
Cualitativamente hablando: ¿Qué hace cada filtro?. 



\question[30 (40)] {\bf{Variabilidad estelar (\verb"variabilidad.ipynb")}}

Datos de la intensidad de una estrella variable RR-Lyrae se encuentran
en \verb"RR_Lyrae_template.dat". 

\begin{itemize}
\item[a)] (10 puntos) Escriba un programa en Python que calcule la
  misma curva de intensidad cuando se toman en cuenta $N$
  componentes de Fourier. 
\item[b)] (10 (15) puntos) Prepare gr\'aficas de la curva
  reconstru\'ida con $N\leq 10$ componentes (i.e. deben aparecer 10
  curvas diferentes).   
\item[c)] (10 (15) puntos)
Prepara una gr\'afica de $\chi^2$ en funci\'on del n\'umero de
componentes $N$ tomadas en cuenta al momento de hacer la
reconstrucci\'on.   
\end{itemize}


\question[40 (50)] {\bf{C\'irculos (\verb"circulos.ipynb")}}

En el archivo \verb"BAO.dat" se encuentran posiciones en un plano
$x-y$. Estos puntos corresponden a la superposici\'on de diferentes
c\'irculos m\'as un fondo de puntos distribuidos aleatoriamente a
partir de una distribuci\'on homog\'enea. 

Encuentre el di\'ametro de estos c\'irculos a partir de m\'etodos que
utilicen la transformada de Fourier. 

Ayuda:
Funci\'on de autocorrelaci\'on \\ 
\verb"http://mathworld.wolfram.com/Autocorrelation.html"\\
\verb"http://mathworld.wolfram.com/Wiener-KhinchinTheorem.html"

\end{questions}
\end{document}
