%--------------------------------------------------------------------
%--------------------------------------------------------------------
% Formato para los talleres del curso de Métodos Computacionales
% Universidad de los Andes
% 2015-20
%--------------------------------------------------------------------
%--------------------------------------------------------------------

\documentclass[11pt,letterpaper]{exam}
\usepackage[utf8]{inputenc}
%\usepackage[spanish]{babel}
\usepackage{graphicx}
\usepackage{tabularx}
\usepackage[absolute]{textpos} % Para poner una imagen completa en la portada
\usepackage{hyperref}

\newcommand{\base}[1]{\underline{\hspace{#1}}}
\boxedpoints
\pointname{ pt}

\extraheadheight{-0.15in}

\newcommand\upquote[1]{\textquotesingle#1\textquotesingle} % To fix straight quotes in verbatim

\begin{document}
\begin{center}
{\Large Métodos Computacionales} \\
Taller 7 \\
Profesor: Jaime Forero\\
Fecha de Publicación: {\small \it Noviembre 10 del 2015}\\
\end{center}

\begin{textblock*}{40mm}(10mm,20mm)
  \includegraphics[width=3cm]{logoUniandes.png}
\end{textblock*}

\begin{textblock*}{40mm}(161mm,20mm)
  \includegraphics[width=3cm]{logoUniandes.png}
\end{textblock*}

\vspace{0.5cm}

{\Large Instrucciones de Entrega}\\

\noindent
La solución a este taller debe subirse por SICUA antes de las 10:00AM
del viernes 21 de Noviembre del 2015. 
\noindent
Si la soluci\'on est\'a en SICUA
antes de las 8:30AM del lunes 16 de Noviembre del 2015 se calificar\'a
el taller sobre 140 puntos. 
\noindent
Los c\'odigos deben estar en un \'unico repositorio de github con un
\'ultimo commit hecho antes de la fecha l\'imite de entrega. El
repositorio debe tener dos carpetas de nombre \verb"solar" y
\verb"poblaciones" para cada uno de los ejercicios A SICUA
solamente se debe responder con la direcci\'on del repositorio.


\begin{questions}
\question[50 (70)] {\bf{Ciclo solar}}

Volvemos a un problema conocido por todos.  El n\'umero de manchas
solares mensuales desde 1610. 



Asumamos, como hicimos en el taller 3, que este ciclo puede describirse por la siguiente funci\'on:
\begin{displaymath}
f(t) = a\cos((2\pi/d) t + b) + c
\end{displaymath}
donde $t$ es el tiempo en a\~nos, y $a$, $b$ ,$c$ y $d$ son par\'ametros libres.
\begin{itemize}

\item (30 (40) puntos) Escriba un programa en C que explore con MCMC
  el espacio de par\'ametros libres para encontrar sus valores m\'as
  probables junto a su incertidumbre para imprimirlos en pantalla.
  El programa debe poder ejecutarse de la siguiente manera 
\begin{verbatim}
./mcmc_solar.x n_steps n_burn
\end{verbatim}
donde \verb"n_steps" es el n\'umero de pasos total que hace la cadena
de markov y \verb"n_burn" es el n\'umero de pasos iniciales de burn
in.

\item (10 (15) puntos) Escriba un programa en Python que grafique las
  densidades de probabilidad para todos los pares de par\'ametros.

\item (10 (15) puntos) Escriba un \verb"makefile" que enlace correctamente
  todos los pasos anteriores.
\end{itemize}




\question[50 (70)] {\bf{Poblaciones}}

Un grupo de bi\'ologos toma datos por casi una d\'ecada de una
poblaci\'on de presas y predadores. Los bi\'ologos intuyen que el
n\'umero de presas $x$ y el numero de predadores $y$ se describe por
un modelo del tipo Lotka-Volterra con las siguientes ecuaciones:

\begin{equation}
\frac{dx}{dt}=x(\alpha - \beta y),
\end{equation}
\begin{equation}
\frac{dy}{dt}=-y(\gamma -\delta x).
\end{equation}

donde $\alpha$, $\beta$, $\gamma$ y $\delta$ son par\'ametors libres
que se quieren buscar a partir de los datos experimentales.

\begin{itemize}
\item (30(40) puntos) Escriba un programa en C que implemente MCMC para
  imprimir en pantalla los valores m\'as probables de estos par\'ametros junto
  con su incertidumbre a partir de los datos en el archivo
  \verb"lotka_volterra_obs.dat"\footnote{El archivo se encuentra en
    \url{https://github.com/ComputoCienciasUniandes/MetodosComputacionalesDatos/tree/master/homework/2014-20/hw_8}}. 

El programa debe poder ejecutarse de la siguiente manera 
\begin{verbatim}
./mcmc_lotkavolterra.x n_steps n_burn
\end{verbatim}
donde \verb"n_steps" es el n\'umero de pasos total que hace la cadena
de markov y \verb"n_burn" es el n\'umero de pasos iniciales de burn
in.

\item (10 (15) puntos) Escriba un programa en Python que grafique las
  densidades de probabilidad para todos los pares de par\'ametros.

\item (10 (15) puntos) Escriba un \verb"makefile" que enlace correctamente
  todos los pasos anteriores.

\end{itemize}


\end{questions}
\end{document}
