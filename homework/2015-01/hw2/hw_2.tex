
%--------------------------------------------------------------------
%--------------------------------------------------------------------
% Formato para los talleres del curso de Métodos Computacionales
% Universidad de los Andes
% 2015-10
%--------------------------------------------------------------------
%--------------------------------------------------------------------

\documentclass[11pt,letterpaper]{exam}
\usepackage[utf8]{inputenc}
\usepackage[spanish]{babel}
\usepackage{graphicx}
\usepackage{mdframed}
\usepackage{tabularx}
\usepackage[absolute]{textpos} % Para poner una imagen completa en la portada
\usepackage{multirow}
\mdfdefinestyle{mystyle}{leftmargin=1cm,rightmargin=1cm,linecolor=red}
\usepackage{float}
\usepackage{hyperref}
\decimalpoint
%\usepackage{pst-barcode}
%\usepackage{auto-pst-pdf}

\newcommand{\base}[1]{\underline{\hspace{#1}}}
\boxedpoints
\pointname{ pt}
%\extrawidth{0.75in}
%\extrafootheight{-0.5in}
\extraheadheight{-0.15in}
%\pagestyle{head}

%\noprintanswers
%\printanswers
\renewcommand{\solutiontitle}{}
\SolutionEmphasis{\color{blue}}

\usepackage{upquote,textcomp}
\newcommand\upquote[1]{\textquotesingle#1\textquotesingle} % To fix straight quotes in verbatim

\begin{document}
\begin{center}
{\Large Métodos Computacionales} \\
Taller 2 - \textsc{C}: Algunas aplicaciones \\
Profesor: Sebastián Pérez Saaibi\\
Fecha de Publicación: {\small \it Febrero 6 de 2015}\\
\end{center}

\begin{textblock*}{40mm}(10mm,20mm)
  \includegraphics[width=3cm]{logoUniandes.png}
\end{textblock*}

\begin{textblock*}{40mm}(161mm,20mm)
  \includegraphics[width=3cm]{logoUniandes.png}
\end{textblock*}

\vspace{0.5cm}

{\Large Fecha de Entrega:  \bf Febrero 20 de 2015 antes de las 11:50AM COT}

\vspace{0.5cm}

{\Large Instrucciones de Entrega}\\

Los scripts de solución de este taller deben ser presentados en un solo archivo con nombre \verb+NombreApellido_HW1.zip+ en \textbf{sicuaplus}. Por ejemplo: yo deber\'ia subir un archivo llamado \verb"SebastianPerez_HW1.zip"

En cada parte del ejercicio se entrega 1/3  de los puntos si el código propuesto es razonable, 1/3 si se puede ejecutar y 1/3 si entrega resultados correctos.


\vspace{0.5cm}


\begin{questions}

\question[50] {\bf{Barajando Arreglos}} En este punto, vamos a aprender a escribir algoritmos para desordenar arreglos numéricos usando \emph{C}.

\begin{parts}
\part[15] Escriba un algoritmo llamado \verb+barajar_n.c+ que desordene un arreglo de \emph{n} elementos de manera aleatoria (puede pensar en una baraja de cartas, por ejemplo). Este algoritmo debe seleccionar un elemento, de manera aleatoria, del arreglo (en el rango \verb+[0, n - 1]+) y después validar si ese elemento ya fue \emph{barajado}.
\part[5] Mejore el anterior algoritmo con la siguiente condición: seleccione un número aleatorio entre \verb+[0, m - 1]+ donde m empieza en n y disminuye en 1 por cada iteración. Llámelo \verb+barajar_m.c+
\part[20] Implemente un \emph{barajador} \verb+O(n)+ así: seleccione un número aleatorio restante (del frente de los números, y ubíquelo en su nuevo lugar en la parte posterior). El elemento sin barajar de la parte posterior, se mueve al frente, donde espera ser barajado. Llámelo \verb+barajar_fy.c+
\part[10] Asuma que se tiene una baraja de 5 cartas. Genere una visualización (usando gnuplot) del funcionamiento de cada algoritmo. Cuál algoritmo es mejor? Comente su respuesta en el código. Llámelo \verb+viz_barajas.c+
\end{parts}
 
\newpage 
 
\question[50] {\bf{Movimiento aparente de las estrellas en el cielo}} Como vemos cada noche las estrellas realizan un movimiento aparente en el cielo, en realidad este movimiento se debe a la rotación de la tierra sobre su propio eje y alrededor del Sol. En el catalogo \verb+hipparcos.csv+ \footnote{\url{https://github.com/spsaaibi/ComputationalToolsData/tree/master/data/Hipparcos}} se encuentra la siguiente información  \verb+id AR DEC Mag Distance+ de cada estrella, Donde  \verb+AR+ y \verb+DEC+ son las coordenadas ecuatoriales \footnote{\url{http://es.wikipedia.org/wiki/Coordenadas_ecuatoriales}} de las estrellas. \verb+Mag+ Esta relacionado con la luminosidad de la estrella y \verb+Distance+ es la distancia a la estrella en parsecs \footnote{\url{http://es.wikipedia.org/wiki/Parsec}}. Todos los codigos deben usar funciones y punteros.


\begin{parts}

\part[15] Escriba un codigo en \verb+C+ que se llame \verb+conversion.c+ que transforme las coordendas del sistema ecuatorial (\verb+AR+ y \verb+DEC+) a las coordenadas horizontales de \verb+Altura+ y \verb+Azimut+. Esto hagalo para todas las latitues y longitudes del mundo. Use la hora de su nacimiento. El codigo debe leer el archivo \verb+hipparcos.csv+ y debe imprimir un archivo \verb+hiparcos_horizontal_lat_lon.csv+ donde \verb+lat+ y \verb+lon+ corresponden a la latitud y longitud usada.    

\part[20] Escriba un codigo que se llame \verb+timelapse.c+ que seleccione las estrellas que er\'an visibles en el firmamento para la latitud y longitud de donde ud nacio. Evolucione el tiempo 5 horas despues de ud haber nacido y grafique el movimiento de las estrellas cada 10 min durante este tiempo. El codigo debe crear un archivo \verb+timelapse.dat+ con los datos usados en la grafica.  

\part[15] Escriba un codigo que lea el archivo \verb+hipparcos_horizontal_lat_lon.csv+ este debe seleccionar las 100 estrellas mas luminosas que estaban en el cielo (\verb+Mag<6.0+) cuando ud nació y grafique en 3D usando coordenadas horizontales estas estrellas en el momento de su nacimiento. El codigo debe crear un archivo \verb+100_hipparcos.dat+ con los datos usados en la grafica.
 
\end{parts}

\end{questions}
\end{document}
