
%--------------------------------------------------------------------
%--------------------------------------------------------------------
% Formato para los talleres del curso de Métodos Computacionales
% Universidad de los Andes
% 2015-10
%--------------------------------------------------------------------
%--------------------------------------------------------------------

\documentclass[11pt,letterpaper]{exam}
\usepackage[utf8]{inputenc}
\usepackage[spanish]{babel}
\usepackage{graphicx}
\usepackage{mdframed}
\usepackage{tabularx}
\usepackage[absolute]{textpos} % Para poner una imagen completa en la portada
\usepackage{multirow}
\mdfdefinestyle{mystyle}{leftmargin=1cm,rightmargin=1cm,linecolor=red}
\usepackage{float}
\usepackage{hyperref}
\decimalpoint
%\usepackage{pst-barcode}
%\usepackage{auto-pst-pdf}

\newcommand{\base}[1]{\underline{\hspace{#1}}}
\boxedpoints
\pointname{ pt}
%\extrawidth{0.75in}
%\extrafootheight{-0.5in}
\extraheadheight{-0.15in}
%\pagestyle{head}

%\noprintanswers
%\printanswers
\renewcommand{\solutiontitle}{}
\SolutionEmphasis{\color{blue}}

\usepackage{upquote,textcomp}
\newcommand\upquote[1]{\textquotesingle#1\textquotesingle} % To fix straight quotes in verbatim

\begin{document}
\begin{center}
{\Large Métodos Computacionales} \\
Taller 1 - \textsc{Unix}: Línea de Comandos \\
Fecha de Publicación: {\small \it Enero 23 de 2015}\\
\end{center}

\begin{textblock*}{40mm}(10mm,20mm)
  \includegraphics[width=3cm]{logoUniandes.png}
\end{textblock*}

\begin{textblock*}{40mm}(161mm,20mm)
  \includegraphics[width=3cm]{logoUniandes.png}
\end{textblock*}

\vspace{0.5cm}

{\Large Fecha de Entrega:  \bf Febrero 6 de 2015 antes de las 11:50AM COT}

\vspace{0.5cm}

{\Large Instrucciones de Entrega}\\

Los 3 scripts de solución de este taller deben ser presentados en un solo archivo con nombre \verb+NombreApellido_HW1.zip+ en \textbf{sicuaplus}. Por ejemplo: yo deber\'ia subir un archivo llamado \verb"SebastianPerez_HW1.zip"

En cada parte del ejercicio se entrega 1/3  de los puntos si el código propuesto es razonable, 1/3 si se puede ejecutar y 1/3 si entrega resultados correctos.


\vspace{0.5cm}

{\Large\bf Ajedrez}\\

En el siguiente repositorio, se pueden encontrar los resultados de juegos de ajedrez históricos (en formato .pgn), actualizados semanalmente \url{http://www.theweekinchess.com/twic}.


\begin{questions}

\question[30] {\bf{Buscando resultados de Ajedrez}} 
 
 Haga un script, llamado \verb+NombreApellido_scraper.sh+ el cual descarge todos los juegos de ajedrez en el siguiente rango de semanas: 21/07/2014 - 19/01/2015 (en formato .pgn\footnote{Portable Game Notation: \url{http://en.wikipedia.org/wiki/Portable_Game_Notation}}). Los archivos tienen que descargarse a una carpeta llamada \verb+resultados_ajedrez/+. IMPORTANTE: Su script se debe encargar de borrar todos los archivos que no sean \verb+.pgn+ dentro de dicha carpeta.
 
 \emph{Idea}: para correr el script, es posible usar el operador \verb+'|'+. Ejemplo: \verb+sh script.sh | ...+
 
 \vspace{0.5 cm}
 
\question[45] {\bf{Haciendo cuentas}}
 En un script llamado \verb+NombreApellido_cuentas.sh+, responda las siguientes preguntas:
\begin{parts}
	\part[10] Cuál es el peso en bytes de todos los archivos?
	\part[5] Cuántos eventos distintos se jugaron en ese periodo? Recuerde que están inidicados con la palabra \verb+Event+.
	\part[15] Cree un archivo \verb+events_desc.txt+ ordenando los eventos por frecuencia descendente.
	\part[15] Patiendo de c), cree un archivo llamado events.csv. Tenga cuidado de eliminar: La palabra \verb+Event+de cada fila, espacios innecesarios en blanco y caracteres especiales (tales como: ",[,]). Ejemplo adecuado del .csv: \verb+504,"German County Youth Ch"+
\end{parts}

\question[25] {\bf{Blancas, Negras o Empate?}} Cree un script llamado \verb+NombreApellido_resultados.sh+ y calcule lo siguiente:

\begin{parts}
\part[20] Número de victorias: blancas, negras, empate, total de juegos. Recuerde que en los datos, se indica el resultado con la palabra \verb+Result+.
\part[5] Haga el mismo cálculo sin usar \verb+grep+ (Puede usar \verb+awk+).
\end{parts}

\end{questions}
\end{document}
