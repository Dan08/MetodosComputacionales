
%--------------------------------------------------------------------
%--------------------------------------------------------------------
% Formato para los talleres del curso de Métodos Computacionales
% Universidad de los Andes
%--------------------------------------------------------------------
%--------------------------------------------------------------------

\documentclass[11pt,letterpaper]{exam}
\usepackage{amsmath}
\usepackage[utf8]{inputenc}
\usepackage[spanish]{babel}
\usepackage{graphicx}
\usepackage{tabularx}
\usepackage[absolute]{textpos} % Para poner una imagen completa en la portada
\usepackage{hyperref}
\usepackage{float}

\newcommand{\base}[1]{\underline{\hspace{#1}}}
\boxedpoints
\pointname{ pt}

\extraheadheight{-0.15in}

\newcommand\upquote[1]{\textquotesingle#1\textquotesingle} % To fix straight quotes in verbatim



\begin{document}
\begin{center}
{\Large M\'etodos Computacionales} \\
Tarea 3 - \textsc{Ecuaciones Diferenciales Ordinarias}\\
12-09-2016\\
\end{center}

\begin{textblock*}{40mm}(10mm,20mm)
  \includegraphics[width=3cm]{logoUniandes.png}
\end{textblock*}

\begin{textblock*}{40mm}(164mm,20mm)
  \includegraphics[width=3cm]{logoUniandes.png}
\end{textblock*}

\vspace{0.3cm}


\noindent
La solución a este taller debe subirse por SICUA antes de las 8:00AM
del jueves 6 de octubre del 2016. 

\noindent
(10 pt) Los c\'odigos deben subirse en un
\'unico archivo \verb".zip" con el nombre
\verb"NombreApellido_hw3.zip". Por ejemplo yo deber\'ia subir el zip
\verb"JaimeForero_hw3.zip". Al descomprimir el zip debe crearse la
carpeta \verb"JaimeForero_hw3" y adentro deben estar los c\'odigos
solicitados.
Ning\'un programa puede utilizar las funciones especiales de numpy,
scipy o scikit-learn para integrar, diferenciar, encontrar ceros,
resolver sistemas de ecuaciones o hacer PCA.

\vspace{0.3cm}

\begin{questions}
\question{\bf{Osciladores acoplados}}

Considere un sistema de $N$ masas atadas por una cuerda como se muestran en la figura \ref{fig:loaded}.

\begin{figure}[H]
  \centering
  \includegraphics[width=10cm]{loaded}
  \caption{\label{fig:loaded} Cuerda cargada.}
\end{figure}

Para separaciones peque\~nas alrededor del punto de equilibrio, la tensi\'on $T$ a lo largo de la cuerda se puede tomar como constante. Seg\'un la figura \ref{fig:part}, las ecuaciones de movimiento est\'an dadas por

\begin{equation*}
\frac{\mathrm{d}y_n^\prime}{\mathrm{d}t} = -\frac{T}{ma}\left(2y_n(t) -y_{n-1}(t) - y_{n+1}(t)\right),
\end{equation*}

\begin{equation*}
y_n^\prime  = \frac{\mathrm{d} y_n}{\mathrm{d} t},
\end{equation*}
%
para $n$ entero tal que $0\leq n\leq N$, con condiciones de frontera fijas $y_0(t) = y_{N+1}(t) = 0$.

\begin{figure}[H]
  \centering
  \includegraphics[width=10cm]{part}
  \caption{\label{fig:part} Secci\'on de la cuerda cargada.}
\end{figure}


Escriba un programa en python (\verb"acoplado.py") que lea un archivo
de texto en el siguiente formato 
\vspace{0.5cm}
\begin{verbatim}
N
T
m
a
y10 y20 y30 ... yN0
y1p0 y2p0 y3p0 ... yNp0
\end{verbatim}
\vspace{0.5cm}
Donde $N$ es el n\'umero de osciladores, $T$, $m$ y $a$ los par\'ametros ya mencionados y las \'ultimas dos l\'ineas representan las condiciones iniciales para las posiciones y las velocidades de cada una de las masas (separadas por espacios).

El programa debe resolver el sistema de ecuaciones para un intervalo
de tiempo que permita visualizar el comportamiento (de tal manera que
no sea menor a un per\'iodo del movimiento ni innecesariamente largo)
y manteniendo las p\'erdidas de energ\'ia por debajo del $10\%$ de la
energ\'ia total inicial.
Al final el programa debe generar:

\begin{itemize}
\item (20 puntos) Un gif animado de nombre \verb"movimiento.gif" la
  evoluci\'on temporal de $y_n$ contra $n$ para todas las masas en
  funci\'on del tiempo.
\item (10 puntos) Una gr\'afica llamada \verb"energia.pdf" de la
  variaci\'on de la ener\'gia total del sistema respecto a la
  energ\'ia inicial, $(E(t) - E_0)/E_0$, como funci\'on del
  tiempo. 
\end{itemize} 
El c\'odigo debe poder ejecutarse como 
\begin{verbatim}
python acoplado.py condiciones.dat
\end{verbatim}
%
donde \verb"condiciones.dat" es el archivo que describe las
condiciones iniciales.



\question{\bf{Ciclos}} \\
La siguiente direcci\'on
\url{https://archive.ics.uci.edu/ml/machine-learning-databases/00360/}
contiene datos correspondientes a mediciones de calidad del aire.
Los datos fueron tomados por Saverio De Vito de la \emph{National Agency for
New Technologies, Energy and Sustainable Economic Development}  en Italia.
El archivo contiene 9358 mediciones de promedios hechos cada hora de
mediciones de una serie de cinco sensores qu\'imicos diferentes. Los
sensores estaban localizados a nivel de suelo en una calle contaminada de
una ciudad italiana. Los datos fueron grabados entre Marzo 2004 y Febrero 2005. 
Una descripci\'on detallada de los datos se encuentra en
\url{https://archive.ics.uci.edu/ml/datasets/Air+Quality}.

(30 puntos) Escriba un programa en python (\verb"ciclos.py") que lea
el archivo \verb"AirQualityUCI.csv" y haga un an\'alisis de Fourier
para obtener dos per\'iodos dominantes (en unidades de d\'ias) que se
encuentren en la variabilidad de los datos de las mediciones reales de
las concentraciones de  
\begin{itemize}
\item CO.
\item Hydrocarburos diferentes a Metano.
\item Benceno.
\item NOx.
\item NO2.
\end{itemize}

Los resultados deben escribirse en un archivo \verb"periodos.dat" que
contiene dos columnas (para cada uno de los per\'iodos) y cinco filas
(para cada uno de las concentraciones).  

El c\'odigo debe poder ejecutarse como 
\begin{verbatim}
python ciclos.py AirQualityUCI.csv
\end{verbatim}


\question{\bf{Mezclas}} \\

Utilizando el mismo conjunto de datos del punto anterior escriba un
c\'odigo en python (\verb"mezclas.py") que haga un an\'alisis PCA y
encuentre las dos primeras componentes principales a partir de las
mismas columnas anteriores. 

El programa debe:
\begin{itemize}
\item (10 puntos) Hacer una gr\'afica \verb"pca.pdf" donde se muestren
  todas las mediciones en el plano correspondiente a las dos variables PCA
  principales. 
\item (10 puntos) Escribir en un archivo \verb"pca.dat" las dos columnas
  correspondientes a las componentes principales.
\item (10 puntos) Escribir en un archivo \verb"varianza.dat" el porcentaje de la
  varianza que explican las dos primeras componentes principales.
\end{itemize}

{\bf Cuidado}. Las diferentes concentraciones no se miden en las mismas
unidades. ¿C\'omo hacer para que las mediciones sean entonces
comparables y las componentes PCA no dependan de las
unidades utilizadas en la medici\'on?

El c\'odigo debe poder ejecutarse como 
\begin{verbatim}
python mezclas.py AirQualityUCI.csv
\end{verbatim}




\end{questions}



\vfill

{\small Universidad de los Andes | Vigilada Mineducación
Reconocimiento como Universidad: Decreto 1297 del 30 de mayo de 1964.
Reconocimiento personería jurídica: Resolución 28 del 23 de febrero de
1949 Minjusticia.}  

\end{document}

