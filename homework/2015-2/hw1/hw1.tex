
%--------------------------------------------------------------------
%--------------------------------------------------------------------
% Formato para los talleres del curso de Métodos Computacionales
% Universidad de los Andes
% 2015, curso de vacaciones
%--------------------------------------------------------------------
%--------------------------------------------------------------------

\documentclass[11pt,letterpaper]{exam}
\usepackage[utf8]{inputenc}
\usepackage[spanish]{babel}
\usepackage{graphicx}
\usepackage{tabularx}
\usepackage[absolute]{textpos} % Para poner una imagen en posiciones arbitrarias
\usepackage{multirow}
\usepackage{float}
\usepackage{hyperref}
\decimalpoint

\begin{document}
\begin{center}
{\Large Métodos Computacionales} \\
Tarea 1 - \textsc{Python b\'asico}\\
2015-08-04.\\
\end{center}

\begin{textblock*}{40mm}(10mm,20mm)
  \includegraphics[width=3cm]{logoUniandes.png}
\end{textblock*}

\begin{textblock*}{40mm}(164mm,20mm)
  \includegraphics[width=3cm]{logoUniandes.png}
\end{textblock*}

\vspace{0.5cm}

\noindent
La solución a este taller debe subirse por SICUA antes de las 8:30AM
del jueves 13 de Agosto del 2015. 
\noindent
Si la soluci\'on est\'a en SICUA
antes de las 8:30AM del s\'abado 8 de Agosto del 2015 se calificar\'a
el taller sobre 125 puntos. 
\noindent
Los tres archivos c\'odigo fuente deben subirse en un \'unico archivo
\verb".zip" con el nombre \verb"NombreApellido_hw1.zip", por ejemplo
yo deber\'ia subir el zip \verb"JaimeForero_hw1.zip".

\vspace{0.5cm}

\begin{questions}
 
\question[20(25)] {\bf{Una varilla sobre una pared}}

Una varilla descansa inclinada sobre una pared y empieza a
deslizarse. Escriba un c\'odigo en Python (\verb"varilla.py") que haga
una gr\'afica de la trayectoria que describe el centro de masa de la
varilla. Puede asumir (aunque no sea f\'isicamente cierto) que la
varilla nunca pierde contacto con la pared. 


\question[30(40)]{\bf{Un c\'irculo que pasa por tres puntos}}

Escriba un programa en Python (\verb"circulo.py") que lea un archivo de
texto con el siguiente formato:

\begin{verbatim}
45.0 23.0
23.0 3.0
34.0 23.0
\end{verbatim}
%
donde cada par de valores representa las coordenadas cartesianas de un
punto para preparar una gr\'afica que muestre los tres puntos y el
c\'irculo que pasa por esos tres puntos. El programa debe detenerse e
imprimir un mensaje de error si el archivo no tiene tres puntos o si
los puntos se encuentran sobre una l\'inea recta. El programa debe
poder ejecutarse como \verb"python circulo.py archivo" donde
\verb"archivo" es el nombre del archivo donde est\'an los puntos. 

\question[50(60)]{\bf{El juego de la vida de Conway}}

El juego de la vida de Conway es el ejemplo m\'as famoso de un
aut\'omata celular
\url{https://en.wikipedia.org/wiki/Conway%27s_Game_of_Life}.

Escriba un c\'odigo en Python (\verb"conway.py") que haga 200
iteraciones del juego de la vida sobre un tablero cuadrado de 500x500
pixeles con condiciones iniciales aleatorias. El resultado debe ser un
GIF animado de nombre \verb"conway.gif".
 


\end{questions}

\end{document}
