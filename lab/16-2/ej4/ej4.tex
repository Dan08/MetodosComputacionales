\documentclass[11pt,letterpaper]{exam}
\usepackage{amsmath}
\usepackage[utf8]{inputenc}
\usepackage[spanish]{babel}
\usepackage{graphicx}
\usepackage{tabularx}
\usepackage[absolute]{textpos} % Para poner una imagen en posiciones arbitrarias
\usepackage{multirow}
\usepackage{float}
\usepackage{hyperref}
\usepackage{breakurl}
\decimalpoint

\begin{document}
\begin{center}

\includegraphics[width=16cm]{header.png}

\vspace{1.0cm}
{\Large Laboratorio de M\'etodos Computacionales - Ejercicio 4} \\
\textsc{Semana 12}\\
2016-II\\
\end{center}

%\begin{textblock*}{40mm}(10mm,20mm)
%  \includegraphics[width=3cm]{logoUniandes.png}
%\end{textblock*}

%\begin{textblock*}{40mm}(164mm,20mm)
%  \includegraphics[width=3cm]{logoUniandes.png}
%\end{textblock*}

\vspace{0.5cm}

\noindent

\vspace{0.5cm}

\begin{questions}
 
\question[2.0] {C}

Escriba un programa en C que realice lo siguiente
\begin{itemize}
\item Calcule la suma de \verb'n' n\'umeros aleatorios, cada uno distribu\'ido uniformemente entre $0$ y $1$.
\item Calcule \verb'm' sumas como la del paso anterior, almacene los resultados en un arreglo y para dicho arreglo imprima el promedio y la desviaci\'on est\'andar separados por un espacio.
\item Realice el paso anterior para \verb'n = 20, 40, 60, 80, 100', imprimiendo en cada paso los promedios y desviaciones est\'andar correspondientes.

La salida del programa se debe ver de la siguiente forma

\begin{verbatim}
ave_20 std_20
ave_40 std_40
ave_60 std_60
ave_80 std_80
ave_100 std_100
\end{verbatim}

donde \verb'ave_n'y \verb'std_n' se refieren al promedio y desviaci\'on est\'andar correspondiente a cada valor de \verb'n'. Esta salida se debe guardar en un archivo.

\question[1.5] {Python}
Escriba un script \verb'.py', que importe el archivo generado en el numeral anterior y realice un \verb'scatter' de desviaci\'on est\'andar vs. promedio. La gr\'afica se debe guardar como una imagen.

\question[1.5] {Make}
Escriba un \verb'makefile' que realice los pasos anteriores (compilar el archivo de C, ejecutarlo correctamente y ejecutar el archivo de Python para generar la imagen), y establezca correctamente las reglas y las dependencias para cada paso. Es decir, s\'olo se deben ejecutar los pasos necesarios seg\'un los archivos que han sido modificados.

\end{questions}

\end{document}
