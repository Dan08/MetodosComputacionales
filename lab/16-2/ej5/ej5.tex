\documentclass[11pt,letterpaper]{exam}
\usepackage{amsmath}
\usepackage[utf8]{inputenc}
\usepackage[spanish]{babel}
\usepackage{graphicx}
\usepackage{tabularx}
\usepackage[absolute]{textpos} % Para poner una imagen en posiciones arbitrarias
\usepackage{multirow}
\usepackage{float}
\usepackage{hyperref}
\usepackage{breakurl}
\decimalpoint

\begin{document}
\begin{center}

\includegraphics[width=16cm]{header.png}

\vspace{1.0cm}
{\Large Laboratorio de M\'etodos Computacionales - Ejercicio 5} \\
\textsc{Semana 14}\\
2016-II\\
\end{center}

%\begin{textblock*}{40mm}(10mm,20mm)
%  \includegraphics[width=3cm]{logoUniandes.png}
%\end{textblock*}

%\begin{textblock*}{40mm}(164mm,20mm)
%  \includegraphics[width=3cm]{logoUniandes.png}
%\end{textblock*}

\vspace{0.5cm}

\noindent

\vspace{0.5cm}

\begin{questions}
 

\question \textbf{Metr\'opolis-Hastings}

El archivo \href{https://github.com/ComputoCienciasUniandes/MetodosComputacionales/blob/master/lab/16-2/ej5/datos.dat}{\textbf{datos.dat}} contiene una serie de datos que siguen la forma de una par\'abola, la primera columna corresponde al eje \verb'x' y la segunda al eje \verb'y'. Escriba un programa en Python o C que realice un ajuste de los datos a la funci\'on.

$$ y(x) = ax^2+bx+c $$
utilizando el algoritmo de Metr\'opolis-Hastings.

El programa debe:

\begin{itemize}
\item Imprimir los valores hallados de los tres par\'ametros \verb'a',\verb'b' y \verb'c'.
\item Graficar los datos y el ajuste en la misma figura.
\end{itemize}

\end{questions}

\end{document}
