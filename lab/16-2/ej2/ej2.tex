\documentclass[11pt,letterpaper]{exam}
\usepackage{amsmath}
\usepackage[utf8]{inputenc}
\usepackage[spanish]{babel}
\usepackage{graphicx}
\usepackage{tabularx}
\usepackage[absolute]{textpos} % Para poner una imagen en posiciones arbitrarias
\usepackage{multirow}
\usepackage{float}
\usepackage{hyperref}
\usepackage{breakurl}
\decimalpoint

\begin{document}
\begin{center}

\includegraphics[width=16cm]{header.png}

\vspace{1.0cm}
{\Large Laboratorio de M\'etodos Computacionales - Ejercicio 2} \\
\textsc{Semana 6}\\
2016-II\\
\end{center}

%\begin{textblock*}{40mm}(10mm,20mm)
%  \includegraphics[width=3cm]{logoUniandes.png}
%\end{textblock*}

%\begin{textblock*}{40mm}(164mm,20mm)
%  \includegraphics[width=3cm]{logoUniandes.png}
%\end{textblock*}

\vspace{0.5cm}

\noindent

\vspace{0.5cm}

\begin{questions}
 
\question[5.0] {}

El alcance horizontal m\'aximo $R$ en un tiro parab\'olico de velocidad inicial $v_0$ y \'angulo $\theta$ es

\begin{equation*}
R = \frac{2v_0^2\sin \theta \cos \theta}{g},
\end{equation*}
donde $g$ es la constante de atracci\'on gravitacional.

Escriba un script en python que halle el \'angulo al cual el alcance horizontal es m\'aximo. Verifique la respuesta para $7$ valores de \verb'guess' diferentes distribu\'idos entre entre $0$ y $\pi/2$.

La salida del programa se debe ver de la siguiente manera

\begin{verbatim}
18.0 44.7135210717
27.0 44.7135215498
36.0 44.7135218142
45.0 44.7135211024
54.0 44.7135866099
63.0 44.7135203491
72.0 44.7134451264
\end{verbatim}

La primera columna el \verb'guess' inicial para el \'angulo y la segunda es el \'angulo de alcance m\'aximo, ambas medidas en grados (No deben utilizar ni obtener exactamente los mismos valores, es s\'olo para dar una idea de c\'omo debe verse).

\end{questions}

\end{document}
